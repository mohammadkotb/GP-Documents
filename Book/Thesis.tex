%%%%%%%%%%%%%%%%%%%%%%%%%%%%%%%%%%%%%%%%%
% Thesis LaTeX Template
%
% This template has been downloaded from:
% http://www.latextemplates.com
%
% Original authors:
% Steven Gunn 
% http://users.ecs.soton.ac.uk/srg/softwaretools/document/templates/
% and
% Sunil Patel
% http://www.sunilpatel.co.uk/thesis-template/
%
% Note:
% Make sure to edit document variables in the Thesis.cls file
%
%%%%%%%%%%%%%%%%%%%%%%%%%%%%%%%%%%%%%%%%%

%----------------------------------------------------------------------------------------
%	PACKAGES AND OTHER DOCUMENT CONFIGURATIONS
%----------------------------------------------------------------------------------------

\documentclass[11pt, a4paper, oneside]{Thesis} % Paper size, default font size and one-sided paper

\graphicspath{{./Pictures/}} % Specifies the directory where pictures are stored
\usepackage[square, numbers, comma, sort&compress]{natbib} % Use the natbib reference package - read up on this to edit the reference style; if you want text (e.g. Smith et al., 2012) for the in-text references (instead of numbers), remove 'numbers' 
\usepackage{multirow}
\usepackage{url}
\usepackage{longtable}
\usepackage{float}
\hypersetup{urlcolor=blue, colorlinks=true} % Colors hyperlinks in blue - change to black if annoying
\title{\ttitle} % Defines the thesis title - don't touch this

\begin{document}

\frontmatter % Use roman page numbering style (i, ii, iii, iv...) for the pre-content pages

\setstretch{1.3} % Line spacing of 1.3

% Define the page headers using the FancyHdr package and set up for one-sided printing
\fancyhead{} % Clears all page headers and footers
\rhead{\thepage} % Sets the right side header to show the page number
\lhead{} % Clears the left side page header

\pagestyle{fancy} % Finally, use the "fancy" page style to implement the FancyHdr headers

\newcommand{\HRule}{\rule{\linewidth}{0.5mm}} % New command to make the lines in the title page

%----------------------------------------------------------------------------------------
%	TITLE PAGE
%----------------------------------------------------------------------------------------

\begin{titlepage}
\begin{center}

\textsc{\LARGE \univname}\\[1.5cm] % University name
\textsc{\Large Graduation Project submitted in partial fulfillment of the B. Sc. Degree}\\[0.5cm] % Thesis type

\HRule \\[0.4cm] % Horizontal line
{\huge \bfseries \ttitle}\\[0.4cm] % Thesis title
\HRule \\[1.5cm] % Horizontal line
 
\begin{minipage}{0.4\textwidth}
\begin{flushleft} \large
\emph{Authors:}\\
{\authornames} % Author name - remove the \href bracket to remove the link
\end{flushleft}
\end{minipage}
\begin{minipage}{0.45\textwidth}
\begin{flushright} \large
\emph{Supervisors:} \\
{\supname} % Supervisor name - remove the \href bracket to remove the link  
\end{flushright}
\end{minipage}\\[3cm]
  
{\large \today}\\[4cm] % Date
%\includegraphics{Logo} % University/department logo - uncomment to place it
 
\vfill
\end{center}

\end{titlepage}

%----------------------------------------------------------------------------------------
%	QUOTATION PAGE
%----------------------------------------------------------------------------------------

%\pagestyle{empty} % No headers or footers for the following pages

%\null\vfill % Add some space to move the quote down the page a bit

%\textit{``Thanks to my solid academic training, today I can write hundreds of words on %virtually any topic without possessing a shred of information, which is how I got a good job %in journalism."}

%\begin{flushright}
%Dave Barry
%\end{flushright}

%\vfill\vfill\vfill\vfill\vfill\vfill\null % Add some space at the bottom to position the quote just %right

%\clearpage % Start a new page

%----------------------------------------------------------------------------------------
%	ABSTRACT PAGE
%----------------------------------------------------------------------------------------

\addtotoc{Abstract} % Add the "Abstract" page entry to the Contents

\abstract{\addtocontents{toc}{\vspace{1em}} % Add a gap in the Contents, for aesthetics

The invention of email permanently changed the way communication take place. 
Email messages provide fast, free and reliable method for communication. 
That's why email has become the most frequently used method for communication 
nowadays.

With the increasing popularity of emails, users spend a significant portion of 
their time reading and replying to emails in their inbox folder. One solution 
is to organize emails into several folders -one folder for each topic- rather 
than one big congested folder which is the inbox. This facilitates the process 
of managing, prioritizing and facilitating the search process for emails. 
The problem is that the process of email categorization has to be done manually, 
which tends to be a tedious and time consuming for most users.

``Smart Email'' is an application addressing these problems by providing automatic email categorization into folders based on machine learning and text mining classification approaches.

\clearpage % Start a new page

%----------------------------------------------------------------------------------------
%	ACKNOWLEDGEMENTS
%----------------------------------------------------------------------------------------

%\setstretch{1.3} % Reset the line-spacing to 1.3 for body text (if it has changed)

%\acknowledgements{\addtocontents{toc}{\vspace{1em}} % Add a gap in the Contents, for %aesthetics

%The acknowledgements and the people to thank go here, don't forget to include your project %advisor\ldots
%}
%\clearpage % Start a new page

%----------------------------------------------------------------------------------------
%	LIST OF CONTENTS/FIGURES/TABLES PAGES
%----------------------------------------------------------------------------------------

\pagestyle{fancy} % The page style headers have been "empty" all this time, now use the "fancy" headers as defined before to bring them back

\lhead{\emph{Contents}} % Set the left side page header to "Contents"
\tableofcontents % Write out the Table of Contents

\lhead{\emph{List of Figures}} % Set the left side page header to "List of Figures"
\listoffigures % Write out the List of Figures

\lhead{\emph{List of Tables}} % Set the left side page header to "List of Tables"
\listoftables % Write out the List of Tables

%----------------------------------------------------------------------------------------
%	ABBREVIATIONS
%----------------------------------------------------------------------------------------

\clearpage % Start a new page

\setstretch{1.5} % Set the line spacing to 1.5, this makes the following tables easier to read

\lhead{\emph{List of Acronyms} % Set the left side page header to "Abbreviations"
}

\listofsymbols{ll} % Include a list of Abbreviations (a table of two columns)
{


% Chapter 5 abbreviations
\textbf{SDD} & \textbf{S}oftware \textbf{D}esign \textbf{D}escription \\
\textbf{TBD} & \textbf{T}o \textbf{B}e \textbf{D}efined \\
\textbf{IMAP} & \textbf{I}nternet \textbf{M}essage \textbf{A}ccess \textbf{P}rotocol \\
\textbf{WEKA} & \textbf{W}aikato \textbf{E}nvironment for \textbf{K}nowledge \textbf{A}nalysis \\
\textbf{MOA} & \textbf{M}assive \textbf{O}nline \textbf{A}nalysis



% \textbf{LAH} & \textbf{L}ist \textbf{A}bbreviations \textbf{H}ere \\
% \textbf{Acronym} & \textbf{W}hat (it) \textbf{S}tands \textbf{F}or \\
}

%----------------------------------------------------------------------------------------
%	PHYSICAL CONSTANTS/OTHER DEFINITIONS
%----------------------------------------------------------------------------------------

%\clearpage % Start a new page

%\lhead{\emph{Physical Constants}} % Set the left side page header to "Physical Constants"

%\listofconstants{lrcl} % Include a list of Physical Constants (a four column table)
%{
%Speed of Light & $c$ & $=$ & $2.997\ 924\ 58\times10^{8}\ \mbox{ms}^{-\mbox{s}}$ %(exact)\\
% Constant Name & Symbol & = & Constant Value (with units) \\
%}

%----------------------------------------------------------------------------------------
%	SYMBOLS
%----------------------------------------------------------------------------------------

%\clearpage % Start a new page

%\lhead{\emph{Symbols}} % Set the left side page header to "Symbols"

%\listofnomenclature{lll} % Include a list of Symbols (a three column table)
%{
%$a$ & distance & m \\
%$P$ & power & W (Js$^{-1}$) \\
% Symbol & Name & Unit \\

%& & \\ % Gap to separate the Roman symbols from the Greek

%$\omega$ & angular frequency & rads$^{-1}$ \\
% Symbol & Name & Unit \\
%}

%----------------------------------------------------------------------------------------
%	DEDICATION
%----------------------------------------------------------------------------------------

%\setstretch{1.3} % Return the line spacing back to 1.3

%\pagestyle{empty} % Page style needs to be empty for this page

%\dedicatory{For/Dedicated to/To my\ldots} % Dedication text

%\addtocontents{toc}{\vspace{2em}} % Add a gap in the Contents, for aesthetics

%----------------------------------------------------------------------------------------
%	THESIS CONTENT - CHAPTERS
%----------------------------------------------------------------------------------------

\mainmatter % Begin numeric (1,2,3...) page numbering

\pagestyle{fancy} % Return the page headers back to the "fancy" style

% Include the chapters of the thesis as separate files from the Chapters folder
% Uncomment the lines as you write the chapters

% Chapter 1

\chapter{Introduction} % Main chapter title

\label{Chapter1} % For referencing the chapter elsewhere, use \ref{Chapter1} 

\lhead{Chapter 1. \emph{Introduction}} % This is for the header on each page - perhaps a shortened title

%----------------------------------------------------------------------------------------
\section{Motivation}

The invention of email permanently changed the way communications take place. Email messages provide fast, free and reliable method for communication. That's why email has become the most frequently used method for communication nowadays.

With the increasing popularity of emails, users spend a significant portion of their time reading and replying to emails in their inbox folder . One solution is to organize emails into several folders -one folder for each topic- rather than one big congested folder which is the inbox. This facilitates the process of managing, prioritizing and facilitating the search process for emails. The problem is that the process of email categorization has to be done manually , which tends to be a tedious and time consuming for most users.

Several machine learning techniques have been applied for solving different email related problems such as spam detection. In our work, the problem of automatic email categorization into folders is addressed, that is, assigning email messages into user-specific folders automatically. This task provides a valuable addition to email clients by avoiding mailboxes cluttering and providing a better automated way for managing and structuring of incoming email messages.

	Email classification introduces new challenges not found in ordinary text classification. This is related to the special nature of email messages. Email messages tend to be small in size compared to ordinary text documents. User created folders don’t necessarily correspond to a single semantic topic, folders might contain email messages for unfinished todos, or emails based on a certain sender or recipient. Email message also arrive in a stream over time which causes more difficulties because the main topic may drift over time.

	Smart Email was created to address those problems. Smart Email will train on the existing user folders and try to automatically classify new incoming emails into the best folder they could belong to. Smart Email works as a web service that keep monitoring the user email account for new incoming emails that need to be classified. The web service allows each user to register with one or more email accounts and provides the user with a central place to monitor and control the classification procedure for his emails.

\section{Scope of Work}

In the previous section we have introduced the problem of automatic email categorization into folders.Smart Email is an application that provides a solution for this problem.

Users receive new emails every day , So Smart Email monitors the user's account and triggers the classification procedure on any new emails. The Smart Email main server downloads the new email and perform the classification procedure to know the best category for this new email. The new Email is moved to the predicted category folder, when the user opens his mail account he will notice the new changes right away. No raw emails is stored at the server side after the classification procedure has taken place.

The user provides feedback on the performance of the classification so that the classifier can re-learn from the misclassified emails. The server side maintains a classification model for each user which can be updated and improved incrementally based on the user feedback.

	Smart email also enables the user to register with multiple email addresses and provides him with a central view to monitor and control everything related to the email classification procedure.


\section{Organization of the Report}

The report is organized into 6 chapters. In chapter 2, a background about email classification techniques will be discussed; related work which tried to solve the problem will be presented in section 1.1 will also be presented. In chapter 3, the proposed email classifier , its main features and a comparison with the related work will be introduced. The project conceptualization will be introduced in chapter 4 together with the proposed workflow for the project. Chapter 5 will include the architecture and design for the Smart Email. Chapter 6 will include the performance evaluation for the email classifier.

% Chapter 2
\newenvironment{my_itemize}
{\begin{itemize}
  \setlength{\itemsep}{0cm}
  \setlength{\parskip}{0cm}}
{\end{itemize}}
\newenvironment{my_enumerate}
{\begin{enumerate}
  \setlength{\itemsep}{0cm}
  \setlength{\parskip}{0cm}}
{\end{enumerate}}

\chapter{Email Classification and Related Work} % Main chapter title

\label{Chapter2} % For referencing the chapter elsewhere, use \ref{Chapter2} 

\lhead{Chapter 2. \emph{Email Classification and Related Work}} % This is for the header on each page - perhaps a shortened title

%----------------------------------------------------------------------------------------

\section{Introduction}
In chapter 1, a general description of the problem was presented, the motivation to develop Smart Email, and the scope of work were introduced.

In this chapter, a background about text and email classification is introduced in
sections 2.2. In section 2.3, applications of email categorization is presented. The machine learning approach for text classification is discussed in section 2.4. In section 2.5, construction of text classifier and different types of classifiers are presented. In section 2.6 related work to Smart Email is presented with features of each. Then, the need to
extend related work is explained in section 2.7.

%==============================================================================

\section{Machine Learning in Automated Text Categorization}
\subsection{Definition of Text Categorization}
Text categorization is the task of assigning a Boolean value to each pair $\langle$ d$_{j}$,c$_{i}$ $\rangle$ $\in$ D X C,  where D is a domain of documents and C = \{c$_{1}$, . . . , c $_{|C|}$\} is a set of predefined categories. A value of T assigned to $\langle$ d$_{j}$,c$_{i}$ $\rangle$ indicates a decision to file d$_{j}$ under c$_{i}$, while a value of F indicates a decision not to file d$_{j}$ under c$_{i}$. More formally, the task is to approximate the unknown target function $\theta$ : D ×C $\rightarrow$ \{T, F\} (that describes how documents ought to be classified) by means of a function $\phi$ : D × C $\rightarrow$ \{T, F\} called the classifier (aka rule, or hypothesis, or model ) such that $\theta$ and $\phi$ coincide as much as possible.\cite{Sebastiani2002}

\subsection{Single-label vs Multi-label Text Categorization}
Classification is a very important topic within supervised learning field. Although the most popular task for classification usually deals with single-label datasets, where every example
is associated with a single label $\lambda$ from a set of disjoint labels L, the multi-label datasets are emerging and gaining interest due to their increasing application to real problems. Multilabel datasets are used when the examples are associated with a set of labels Y $\subseteq$ L, as occurs with email classification, image annotation, or genomics.

Two main tasks can be defined when learning from multilabel
data: 
\begin{itemize}
\item Multi-label classification (MLC) that returns a subset of labels to be associated with a given example (it can be considered as a bipartition of the label set considering relevant and irrelevant elements);
\item label ranking (LR) that returns an ordering of the labels according to their relation with the example. \cite{Carmona2011}
\end{itemize}

\subsection{Category Pivoted vs Document Pivoted Text Categorization}
There are two different ways of using a text classifier. Given d$_{j}$ $\in$ D, we might want
to find all the c$_{i}$ $\in$ C under which it should be filed (document-pivoted categorization
– DPC); alternatively, given c$_{i}$ $\in$ C, we might want to find all the d$_{j}$ $\in$ D that should be filed under it (category-pivoted categorization – CPC).
DPC is thus suitable when documents become available at different moments in time, e.g. in filtering e-mail. CPC is instead suitable when: 
\begin{itemize}
\item a new category c$_{|C|+1}$ may be added to an existing set C = \{c$_{1}$, . . . , c$_{|C|}$\} after a number of documents have already been classified under C;
\item these documents need to be reconsidered for classification under c$_{|C|+1}$.
\end{itemize}. DPC is used more often than CPC, as the former situation is more common than the latter. \cite{Sebastiani2002}

\section{Applications of Email Categorization}
\subsection{Spam Detection}
Spam remains a serious problem today because it continues to be a very profitable business for spammers. Spam takes on various forms from adult content, selling products/services, pharmaceuticals to stock promotions, job offers, etc.

Applying machine learning techniques for spam email detections plays a major role in protecting end users from spam email messages.\cite{peifeng2007}

\subsection{Email Organization}
Email has been an efficient and popular communication mechanism as the number of
Internet users’ increases. Therefore, email management has become an important and growing problem for individuals and organizations because it is prone to misuse. One of the problems that are most paramount is disordered email message, congested and un-structured emails in mail boxes.

It may be very hard to find archived email message, search for previous emails with specified contents or features when the mails are not well structured and organized.

At this stage new effective method for managing information in email, reducing email overloads is developed by classifying emails based on important words. \cite{taiwo2007}

\section{The Machine Learning Approach for Text Classification}

In the 80s the most popular approach (at least in operational settings) for the
creation of automatic document classifiers consisted in manually building, by means
of knowledge engineering (KE) techniques, an expert system capable of taking Text Categorization (TC) decisions. Such an expert system would typically consist of a set of manually defined logical rules.

Since the early 90s, the Machine Learning (ML) approach to TC has gained popularity and has eventually become the dominant one. In ML terminology, the classification problem is an activity of supervised learning, since the learning process is 'supervised' by the knowledge of
the categories and of the training instances that belong to them.

The advantages of the ML approach over the KE approach are evident. The engineering effort goes towards the construction not of a classifier, but of an automatic builder of classifiers (the learner ). \cite{Sebastiani2002}

\subsection{Training Set, Test Set, and Validation Set}
The ML approach relies on the availability of an initial corpus = \{d$_{1}$, . . . , d$_{|\Omega|}$\} $\subset$ D of documents preclassified under C = \{c$_{1}$, . . . , c$_{|C|}$\}. That is, the values of the total function $\theta$ : D X C $\rightarrow$ \{T, F\} are known for every pair $\langle$ d$_{j}$ , c$_{i}$ $\rangle$ $\in$   $\Omega$ X C. A document d$_{j}$ is a positive example of c$_{i}$ if (d$_{j}$ , c$_{i}$) = T , a negative example of c$_{i}$ if
(d$_{j}$ , c$_{i}$) = F.

The following subsection represents the different data sets used  in different research papers for email classification:

\subsubsection{Datasets used in Email Classification}
\subparagraph{Enron Dataset}
    \begin{my_itemize}
        \item Automatic Categorization of Email into Folders \cite{RON04}
        \item An Object Oriented Email Clustering Model Using  Weighted Similarities 
  between Email Attributes \cite{NARESH10}
        \item Using GNUsmail to Compare Data Stream Mining Methods \cite{JOSE11}
    \end{my_itemize}
\subparagraph{SRI Dataset}
    \begin{my_itemize}
        \item Automatic Categorization of Email into Folders \cite{RON04}
    \end{my_itemize}
\subparagraph{Pine Dataset}
    \begin{my_itemize}
        \item Email classification for contact centers \cite{ANI03}
    \end{my_itemize}
\subparagraph{Private Dataset}
    \begin{my_itemize}
        \item Enterprise Email Classification Based on Social Network Features \cite{MIN11}
        \item E-Classifier: A Bi-Lingual Email Classification System \cite{NOUF08} 
        \item Automatically tagging email by leveraging other users folders \cite{YEHUDA11}
    \end{my_itemize}
\subparagraph{Public Pua}
    \begin{my_itemize}
        \item Email Categorization Using Multi-Stage Classification Technique \cite{MD07}
    \end{my_itemize}

\section{Construction of Text Classifier}
The construction of text classifiers has been tackled in a variety of ways.
In this section we will deal only with the methods that have been most popular in text classification. We start by discussing the general form that a text classifier has.  In subsections 2.5.1 to 2.5.3 we discuss number of approaches that have been applied in the text classification literature. In general we will assume. The presentation of the algorithms will focus on the methods for classifier learning rather than on the effectiveness and efficiency of the classifiers built by means of them.\cite{Sebastiani2002}

\subsection{Probabilistic classifiers}
The construction of a ranking classifier for category c$_{i}$ $\in$ C usually
consists in the definition of a function CSV$_{i}$ : D $\rightarrow$ [0, 1] that, given a document d$_{j}$ , returns a categorization status value for it, i.e. a number between 0 and 1 that, roughly speaking, represents the evidence for the fact that d$_{j}$ $\in$ c$_{i}$.

Probabilistic classifiers view CSV$_{i}$(d$_{j}$) in terms of $P(c_{i} | d_{j})$, i.e. the probability that a document d$_{j}$ belongs to c$_{i}$, and compute this probability by an application of Bayes’ theorem, given by \[ P(c_{i}|d_{j}) = \frac{P(c_{i})P(d_{j}|c_{i})}{P(d_{j})} \]

In order to alleviate this problem it is common to make the assumption that any two coordinates of the document vector are, when viewed as random variables, statistically independent of each other; this independence assumption is encoded by the equation
\[ P(d_{j}|c_{i}) = \prod_{k=1}^{|\tau|} P(w_{kj}|c_{i}) \]

Probabilistic classifiers that use this assumption are called Naive Bayes classifiers, and account for most of the probabilistic approaches to TC in the literature.

\subsection{Building classifiers by support vector machines}
The support vector machine (SVM) method has been introduced in TC by Joachims [1998, 1999] and subsequently used in [Drucker et al. 1999; Dumais et al. 1998; Du- mais and Chen 2000; Klinkenberg and Joachims 2000; Taira and Haruno 1999;

Yang and Liu 1999]. In geometrical terms, it may be seen as the attempt to
find, among all the surfaces $\sigma 1$,$ \sigma 2$, ... in $|T|$ -dimensional
space that separate the positive from the negative training examples (decision
        surfaces), the $\sigma i$ that separates the positives from the negatives by the widest possible margin, i.e. such that the separation property is invariant with respect to the widest possible traslation of $\sigma i$ .

This idea is best understood in the case in which the positives and the
negatives are linearly separable, in which case the decision surfaces are
($|T|$-1)-hyperplanes.

In the 2-dimensional case , various lines may be chosen as decision surfaces. The SVM method chooses the middle element from the “widest” set of parallel lines, i.e. from the set in which the maximum distance between two elements in the set is highest. It is noteworthy that this “best” decision surface is determined by only a small set of training examples, called the support vectors.

\subsection{On-line Methods}
Methods for learning linear classifiers are often partitioned in two broad classes, batch methods and on-line methods.

Batch methods build a classifier by analysing the training set all at once. On-line (aka incremental) methods build a classifier soon after examining the first training document, and incrementally refine it as they examine new ones. This may be an advantage in the applications in which the 'meaning' of the category may change in time, as e.g. in adaptive filtering. This is also suitable for applications (e.g. semi-automated classification, adaptive filtering) in which we may expect the user of a classifier to provide feedback on how test documents have been classified, as in this case further training may be performed during the operating phase by exploiting user feedback.

\section{Related Work}

\subsection{POPFile}
POPFile is a free, open source, cross-platform mail filter originally written in Perl by John Graham-Cumming and maintained by a team of volunteers. It uses a naive Bayes classifier to filter mail. This allows the filter to "learn" and classify mail according to the user's preferences. Typically it is used to filter spam mail. It can also be used to sort mail into other user defined "buckets" or categories - for example, the user may define a bucket into which work email is sorted.

The program works in several different modes. In the most popular mode, it sets itself up as a proxy between the email client and the POP3 server. As mail is downloaded via POP3, the filter identifies and classifies mail and makes a user defined modification to the subject line, appending the name of the appropriate bucket. The user then sets up rules in the mail client to sort the mail based on the subject line modification. An HTML based interface can be used to instruct POPFile, allowing users to correct errors in classifications and thus train the system to be sensitive to the user's specific requirements.

As an alternative to the subject-line modification (or as a supplement to it), the system can also be configured to use custom mail headers instead.

In another possible mode, POPFile can work as an IMAP client that monitors an IMAP server for incoming mail and also for messages moved by the user. Incoming emails are categorized and then immediately moved to the folder corresponding to the categorization. To train POPFile in this mode, the user only needs to move the message to the correct folder, i.e. to the folder where POPFile should have moved the message.

\subsection{Gmail Priority Inbox}
Gmail priority inbox is an algorithmic-based filtering system that predicts which messages are most important to the recipient and displays them atop the inbox queue.

Gmail uses a variety of signals to identify important email, including which messages the user opens and which messages he replies to.

\subsection{Classification according to the different learning algorithms used in different 
papers}
The following table classifies some recent research papers in the field of email classification according to the different learning algorithms for email classification.
\begin{center}
\begin{tabular}{|p{2cm}|p{2cm}|p{2cm}|p{2cm}|p{2cm}|p{2cm}|}
\hline
\multicolumn{6}{|c|}{Learning Algorithm} \\
\hline
SVM & Na\"{\i}ve Bayes & Neural Networks & Max. Entropy / Winnow & Nnge / Hoeffing Trees & Graph Mining \\ \hline
Email Classification with Co-training \cite{SVETLANA01} &
Email Classification with Co-training \cite{SVETLANA01} &
Email Classification: Solution with Back Propagation Technique \cite{mous05} & 
Automatic Categorization of Emails into Folders \cite{RON04} &
Using GNUsmail to compare Data Stream Mining Methods for On-line Email Classification \cite{JOSE11} &
A graph Based Approach for Multi-Folder Email Classification \cite{sift02} \\ \hline

Automatic Categorization of Emails into Folders \cite{RON04} &
Automatic Categorization of Emails into Folders \cite{RON04} &
Email Classification Using Semantic Feature Space \cite{YUN08} & 
&
& \\ \hline
\end{tabular}
\end{center}
\newpage

\subsection{Classification according to the different learning capabilities}
The following table classifies some recent research papers in the field of email classification according to the different learning capabilities (on-line or off-line) for email classification.

\begin{center}
\begin{tabular}{|p{6cm}|p{6cm}|}
\hline
\multicolumn{2}{|c|}{Learning Capability} \\
\hline
On-line Learning & Off-Line Learning 
\\ \hline
Using GNUsmail to Compare Data Stream Mining Methods \cite{JOSE11} &
An Object Oriented Email Clustering Model Using  Weighted Similarities 
between Email Attributes \cite{NARESH10}
\\ \hline

GNUsmail: Open Framework for On-line Email Classification \cite{MANUEL11}
& Content Based Email Classification System by applying Conceptual Maps \cite{BASKARAN09}
\\ \hline

& E-Classifier: A Bi-Lingual Email Classification System \cite{NOUF08}
\\ \hline

& Email classification for contact centers \cite{ANI03}
\\ \hline

& 
Automatic Categorization of Email into Folders \cite{RON04}

\\ \hline

\end{tabular}
\end{center}



%================================= FEATURES
\subsection{Different features used for email classification}
This section summarizes the different features used for email classification in different research papers
    \subparagraph{Automatic Categorization of Email into Folders \cite{RON04}}
	\begin{my_itemize}
		\item bag-of-words document representation: messages are represented as vectors of word counts.
		\item Words are downcased.
		\item 100 most frequent words and words that appear only once in the training set are removed, and the remaining words are counted in each message to compose a vector.
		\item In future work, richer representations will be considered, including the following:
			\begin{itemize}
				\item Different sections of each email can be treated differently. For example, the system could create distinct features for words appearing in the header, body, signature, attachments, etc.
				\item Named entities may be highly relevant features.
			\end{itemize}
	\end{my_itemize}

    \subparagraph{Email Classifications For Contact Centers \cite{ANI03}}
		\begin{my_itemize}
			\item Feature sets used for experiments included:
				\begin{itemize}
					\item Non-inflected words.
					\item Noun phrases.
					\item Verb phrases.
					\item Punctuation.
					\item Length of the Email.
					\item Dictionaries.
				\end{itemize}
		\end{my_itemize}

	\subparagraph{Using GNUsmail to Compare Data Stream Mining Methods for On-line Email \cite{JOSE11}}
		\begin{my_itemize}
			\item The main feature of the text preprocessing module is a multi-layer filter structure, responsible for performing feature extraction tasks.
			\item The Inbox and Sent folders are skipped in the learning process because they can be thought of as non-specific folders.
			\item Every mail belonging to any other folder (that is, to any topical folder ) goes through a pipeline of linguistic operators which extract relevant features from it.
			\item As the number of possible features is prohibitively large, only the most relevant ones are selected.
		\end{my_itemize}
   
	\subparagraph{Content Based Email Classification System by applying Conceptual Maps \cite{BASKARAN09}}
		\begin{my_itemize}
			\item Unstructured text: consists of fields like the subject and body.
			\item Categorical text: includes fields such as "to" and "from".
			\item Numeric data: includes such features as the message size, number of
recipients and counts of particular characters.
		\end{my_itemize}

\newpage
\subsection{Chronological sort of classification papers}
\subparagraph{2011}
\begin{my_itemize}
  \item Using GNUsmail to Compare Data Stream Mining Methods for On-line Email Classification \cite{JOSE11}
  \item Enterprise Email Classification Based on Social Network Features \cite{MIN11}
  \item Automatically tagging email by leveraging other users folders \cite{YEHUDA11}
\end{my_itemize}

\subparagraph{2010}
\begin{my_itemize}
  \item An Object Oriented Email Clustering Model Using Weighted Similarities between Emails Attributes \cite{NARESH10}
  \item A Graph-Based Approach for Multi-Folder Email Classification \cite{sift02}
\end{my_itemize}

\subparagraph{2009}
\begin{my_itemize}
  \item Content Based Email Classification System by applying Conceptual Maps \cite{BASKARAN09}
  \item Email Classification: Solution with Back Propagation Technique \cite{mous05}
\end{my_itemize}

\subparagraph{2008}
\begin{my_itemize}
  \item A new approach to Email classification using Concept Vector Space Model \cite{CHAO08}
  \item E-Classifier: A Bi-Lingual Email Classification System \cite{NOUF08}
  \item Ontology based classification and categorization of email \cite{BALAKUMAR08}
  \item Applying Machine learning Algorithms for Email Management \cite{mous03}
\end{my_itemize}

\subparagraph{2007}
\begin{my_itemize}
  \item Email Categorization Using Multi-Stage Classification Technique \cite{MD07}
\end{my_itemize}

\subparagraph{2005}
\begin{my_itemize}
  \item An Email Classification Model Based on Rough Set Theory \cite{WENQING05}
  \item eMailSift: Email Classification Based on Structure and Content \cite{sift01}
\end{my_itemize}

\subparagraph{2004}
\begin{my_itemize}
  \item Automatic Categorization of Email into Folders \cite{RON04}
  \item Co-training with a Single Natural Feature Set Applied to Email Classification \cite{mous04}
\end{my_itemize}

\subparagraph{2003}
\begin{my_itemize}
  \item Email Classifications For Contact Centers \cite{ANI03}
\end{my_itemize}


\section{Need To Extend Related Work}
From the research papers related to email classification we conclude the following:
\begin{my_itemize}
    \item SVM Achieved the best results in most papers, but it is computationally expensive and has the hardest implementation.
    \item The basic form of Na\"{\i}ve Bayes algorithm has the simplest implementation but has very low classification accuracy compared to other algorithms.
    \item Online learning techniques are still under development, they are very hard to implement but characterized by their
    ability to classify new coming email without rebuilding the model.
    \item Offline learning techniques are used in most papers.
    \item Enron dataset is the most commonly used.
\end{my_itemize}

In section 2.6 related work was introduced that tries to solve the problem, but most of them targets only enterprises which will not be the case in Smart Email.

\section{Conclusion}
In this chapter, text and email classification were explained and many related techniques and technologies were discussed. Also related applications to Smart Email were introduced.

In the next chapter, the main features used for email classifier will be introduced. These features are based on the related work presented in this chapter.
\label{sec:3_classifier_features}
\label{sec:3_detailed_classifier_features}
\label{sec:3_classifier_tuple}
\label{sec:conclusion_3}
% Chapter 3

\chapter{The Suggested Features Of Smart Email Classifier} % Main chapter title

\label{Chapter3} % For referencing the chapter elsewhere, use \ref{Chapter3} 

\lhead{Chapter 3. \emph{The Suggested Features of Smart Email Classifier}} % This is for the header on each page - perhaps a shortened title

%----------------------------------------------------------------------------------------

\section{Introduction}
In the previous chapter, a background about text and email classification was 
presented. Also, technologies that are related to text classification were 
introduced. Some related applications that are similar to Smart Email with their 
features were discussed.

In this chapter, the suggested features for smart email classifier are discussed.
In section \ref{sec:3_classifier_features}, the features used for email 
classification is presented. Section \ref{sec:3_detailed_classifier_features} 
includes the detailed classifier features. Section \ref{sec:3_classifier_tuple} 
presents the tuple of classifier features used as an input for the email 
classifier. The Chapter is concluded in section \ref{sec:conclusion_3}.
%==============================================================================
\newpage
\section {Classifier Features}
\label{sec:3_classifier_features}
\begin{longtable}{|>{\centering}p{2.5cm}|>{\centering}p{3cm}|>{\centering}p{3cm}|>{\centering}p{3cm}|}
\hline
\multirow{19}{2.5cm}{Features needed for the email classifier (Automatic Categorization
of emails into folders)}
 & \multicolumn{3}{c|}{Features}\tabularnewline
\cline{2-4}
\cline{2-4} 
 & Email & Receiver & Attachment\tabularnewline
\cline{2-4} 
 & Email ID \cite{Anatomy00} & Domain of receiver \cite{Carmona2011} \cite{MANUEL11} &  Attachment  type \cite{Carmona2011} \cite{MANUEL11}\tabularnewline
\cline{2-4} 
 & Body Length \cite{Carmona2011} \cite{MANUEL11} & Number of CC \cite{Carmona2011} \cite{MANUEL11} & Has an attachment \cite{Carmona2011} \cite{MANUEL11}\tabularnewline
\cline{2-4} 
 & Content Type \cite{Anatomy00} & Number of receivers \cite{Carmona2011} \cite{MANUEL11} & Number of attachments \cite{Carmona2011} \cite{MANUEL11}\tabularnewline
\cline{2-4} 
 & Domain of sender \cite{Carmona2011} \cite{MANUEL11} & Number of To \cite{Carmona2011} \cite{MANUEL11} & \tabularnewline
\cline{2-4} 
 & Email Date \cite{KIRI2004} \cite{Anatomy00} & Receiver Username \cite{Carmona2011} \cite{MANUEL11} & \tabularnewline
\cline{2-4} 
 & Email Sender \cite{Carmona2011} \cite{RON04} \cite{Anatomy00} \cite{MANUEL11} &  & \tabularnewline
\cline{2-4} 
 & Email Signature \cite{MANUEL11} &  & \tabularnewline
\cline{2-4} 
 & Email Subject \cite{Carmona2011} \cite{RON04} \cite{MANUEL11} &  & \tabularnewline
\cline{2-4} 
 & Is Bcc \cite{Carmona2011} \cite{RON04} \cite{MANUEL11} &  & \tabularnewline
\cline{2-4} 
 & Is distribution List \cite{Carmona2011} \cite{MANUEL11} &  & \tabularnewline
\cline{2-4} 
 & Language \cite{Carmona2011} \cite{MANUEL11} &  & \tabularnewline
\cline{2-4} 
 & MIME Version \cite{Anatomy00} &  & \tabularnewline
\cline{2-4} 
 & Number of punctuation Letters \cite{Carmona2011} \cite{MANUEL11} &  & \tabularnewline
\cline{2-4} 
 & Percentage of capital letters \cite{Carmona2011} \cite{MANUEL11} &  & \tabularnewline
\cline{2-4} 
 & Sender Username \cite{Carmona2011} \cite{MANUEL11} &  & \tabularnewline
\cline{2-4} 
 & Subject Length \cite{Carmona2011} \cite{MANUEL11} &  & \tabularnewline
\cline{2-4} 
 & Wordgram Frequency \cite{Carmona2011} \cite{RON04} \cite{MANUEL11} &  & \tabularnewline
\hline
\end{longtable}
%----------------------------------------------------------------------
\section {Detailed Classifier Features}
\label{sec:3_detailed_classifier_features}

\begin{longtable}{|>{\centering}p{2cm}|>{\centering}p{2.5cm}|>{\centering}p{3cm}|>{\centering}p{3cm}|>{\centering}p{3cm}|}
\hline 
Category & Feature & Description & Values & Source (preparation)\tabularnewline
\hline
\hline
Email & Email ID \cite{Anatomy00} & Identifier for the email message & Long & System maintained primary key\tabularnewline
\cline{2-5}
 & Domain of sender \cite{Carmona2011} \cite{MANUEL11} & Mail Service Provider (gmail.com, hotmail.com, ..etc) & String & Obtained from the sender's email, by taking the substring after the
'@' character\tabularnewline
\cline{2-5}
 & Language \cite{Carmona2011} \cite{MANUEL11} & Dominant language in the email body & String & Use a special module to detect the language type of the email body\tabularnewline
\cline{2-5}
 & Email Sender \cite{Carmona2011} \cite{RON04} \cite{Anatomy00} \cite{MANUEL11} & Email address of the sender & String & Obtained directly from the email header\tabularnewline
\cline{2-5}
 & Content Type \cite{Anatomy00} & Content type  & String & Obtained directly from the email header\tabularnewline
\cline{2-5}
 & Email Date \cite{KIRI2004} \cite{Anatomy00} & Date of sending the email represented as the number of milliseconds
since January 1, 1970, 00:00:00 GMT & Long & The date is obtained directly from the email header and then transformed
to the long representation\tabularnewline
\cline{2-5}
 & MIME Version \cite{Anatomy00} & MIME is an internet standard to extend the format of the email to
support non-ASCII data & Integer & Obtained directly from the email header\tabularnewline
\cline{2-5}
 & Bcc \cite{Carmona2011} \cite{RON04} \cite{MANUEL11} & List of email receivers as Bcc & Each recipient is represented as a boolean attribute in the feature
tuple. & Obtained directly from the email header \tabularnewline
\cline{2-5}
 & Number of punctuation Letters \cite{Carmona2011} \cite{MANUEL11} & Number of punctuation characters in the body & Integer & Count the number of punctuation letters in the email body\tabularnewline
\cline{2-5}
 & Is distribution List \cite{Carmona2011} \cite{MANUEL11} & Flag to indicate whether the client received this email from a group/distribution
list or not & Boolean & Obtained directly from email header\tabularnewline
\cline{2-5}
 & Email Signature \cite{MANUEL11} & Signature of the email sender, at the end of the email & String & The signature is extracted from email body\tabularnewline
\cline{2-5}
 & Wordgram Frequency \cite{Carmona2011} \cite{RON04} \cite{MANUEL11} & Email Wordgram Frequency & Integer & Count the number of wordgrams in the email\tabularnewline
\cline{2-5}
 & Subject Length \cite{Carmona2011} \cite{MANUEL11} & Length of the email subject & Integer & Calculate the size of the subject string\tabularnewline
\cline{2-5}
 & Percentage of capital letters \cite{Carmona2011} \cite{MANUEL11} & Percentage of the capital letters to the letters in the email body & Double & Count the number of capital letters and divide it by the sum of the
sizes of all ASCII words in the email body\tabularnewline
\cline{2-5}
 & Body Length \cite{Carmona2011} \cite{MANUEL11} & Size of the email body & Integer & Calculate the size of the body string\tabularnewline
\cline{2-5}
 & Sender Username \cite{Carmona2011} \cite{MANUEL11} & Name of Sender & String & Obtained directly from email header\tabularnewline
\cline{2-5}
 & Total Number of words & Number of words in the email body & Integer & Calculate the number of words in the email body\tabularnewline
\cline{2-5}
 & Email Subject \cite{Carmona2011} \cite{RON04} \cite{MANUEL11} & Subject of the email & String & Obtained directly from the email header\tabularnewline
\hline 
Receiver & Receiver Username \cite{Carmona2011} \cite{MANUEL11} & Name of receiver & String & Obtained directly from email header\tabularnewline
\cline{2-5}
 & Number of receivers \cite{Carmona2011} \cite{MANUEL11} & Number of email receivers & Integer & Count the number of receivers obtained from email header\tabularnewline
\cline{2-5}
 & Number of CC \cite{Carmona2011} \cite{MANUEL11} & Number of CC recipients & Integer & Count the number of CC recipients obtained from email header\tabularnewline
\cline{2-5}
 & Number of To \cite{Carmona2011} \cite{MANUEL11} & Number of CC & Integer & Count the number of receivers mentioned in the TO header\tabularnewline
\cline{2-5}
 & Domain of receiver \cite{Carmona2011} \cite{MANUEL11} & Mail Service Provider(s) for the receiver(s) & String & Obtained directly from email header\tabularnewline
\hline 
Attachment & Number of attachments \cite{Carmona2011} \cite{MANUEL11} & Number of attached files in the email  & Integer & Count the number of attachments obtained from the IMAP interface\tabularnewline
\cline{2-5}
 &  Attachment  type \cite{Carmona2011} \cite{MANUEL11} & Type of attachment & String & 	Has an attachment \cite{Carmona2011} \cite{MANUEL11}	Flag to denote whether the email
has an attachment	Boolean 	If the number of attachment is zero, return
false, else return true
\tabularnewline
\hline
\end{longtable}
%----------------------------------------------------------------------

%-------------------------------------------------------------------------
\newpage
\section {Tuple of the classifier Features (Classifier Input)}
\label{sec:3_classifier_tuple}

\begin{center}
\begin{tabular}{|c|}
\hline 
email\_id\tabularnewline
\hline
date\tabularnewline
\hline 
sender\_email\tabularnewline
\hline 
sender\_username\tabularnewline
\hline 
Domain of sender\tabularnewline
\hline 
is\_bbc\tabularnewline
\hline 
subject\tabularnewline
\hline 
Subject\_length\tabularnewline
\hline 
content\tabularnewline
\hline 
content\_mime\_version\tabularnewline
\hline 
body\_length\tabularnewline
\hline 
signature\tabularnewline
\hline 
number\_of\_receivers\tabularnewline
\hline 
Percentage\_of\_capital\_letters\tabularnewline
\hline 
Number\_of\_punctuation\_letters\tabularnewline
\hline 
language\tabularnewline
\hline 
has\_attachments\tabularnewline
\hline 
number\_of\_attachments\tabularnewline
\hline
\end{tabular}
\end{center}

\newpage


\section{Conclusion}
\label{sec:conclusion_3}
In this chapter the main features used for email classification was introduced.
At the end of the chapter, the tuple of the classifier features was presented. In the next chapter, the project requirements and design  will be  discussed.

% Chapter 4
\newenvironment{my_desc}
{\begin{description}
  \setlength{\itemsep}{0cm}
  \setlength{\parskip}{0cm}}
{\end{description}}

\chapter{Smart Email Architecture and Design} % Main chapter title

\label{Chapter4} % For referencing the chapter elsewhere, use \ref{Chapter4} 

\lhead{Chapter 4. \emph{Smart Email Architecture and Design}} % This is for the header on each page - perhaps a shortened title

\section{Introduction}
In chapter 3, the logical schema and the features of the smart email classifier
were presented.

In this chapter, software requirements specification and software design
description were explained in sections 4.2 and 4.3 respectively. The chapter is
concluded in section 4.4.

\section{Software Requirements Specification} % Main chapter title
%==============================================================================
\subsection{Overview}
In this section, a brief overview about the software requirements specification
is shown, such as, project objective, assumptions, limitations, and open items
and risks.

\subsubsection{Project Objective}
The business objective of this project is building a web service that provides
automatic email categorization into user-defined folders based on machine 
learning and data mining classification techniques.

%==============================================================================

\subsubsection{Assumptions}
Some assumptions generally must be made in order to write a brief definition 
of the application. Some assumptions are technical, while others are business natured.  Assumptions 
critical to the success of this project are listed below:
\begin{my_itemize}
  \item each user will have his own trained classification model based on the user supplied training data;
  \item a web service will be implemented to provide email classification support for different email 
	providers (Example: Google, Hotmail, Yahoo);
  \item english language will be supported by default;
  \item the servers providing the web service have to provide sufficient 
	processing power for achieving reasonable response time;
  \item email classification will be based on the email subject and content. 
	Other features can be used as well, such as the email sender and recepient(s). 
	However, email attachments will not be considered as a classification feature for the email.
\end{my_itemize}

%==============================================================================
\subsubsection{Limitations}
Some limitations are assumptions on the extent of feature scope. Others are restrictions on resources 
or methods for achieving the objectives. The limitations of this project are listed below:

\begin{my_itemize}
  \item sufficient number of emails is needed for each email label to achieve 
	a reasonable classification accuracy;
  \item emails with size less than a specified threshold won't be classified with
  high accuracy.
\end{my_itemize}

%==============================================================================
\subsubsection{Open items and Risks}
During the analysis of the project and in the process of writing this document, 
some issues remain open.
\begin{my_itemize}
  \item Multi-label classification.
  \item Arabic support.
  \item Securing the communication between the client and the classification web service.
  \item Improving classification accuracy.
  \item Specifications for the web server providing the web service.
\end{my_itemize}


%==============================================================================
\subsection{Proposed Workflow}
%==============================================================================
\subsubsection{Overview}
This section includes a concise description of the features and operation of the 
proposed solution. The context and workflow of the proposed solution are defined in subsequent sections.

Features:
\begin{my_itemize}
  \item building a classification web service with a REST API \cite{REST};
  \item building an email monitoring web service that sends every new incoming
  email to the classification web service and applies the returned label;
  \item building a browser extension that is injected into the user's email web interface
  and provides an on-demand email classification using the classification web service.
\end{my_itemize}

%==============================================================================
\subsubsection{Context Diagram}
The Context Diagram in figure 4.1 illustrates the modules, business processes, ...etc that feed 
or interact with this proposed application.
\begin{figure}[H]
  \centering
  \includegraphics[width=13cm]{context_diagram.png}
  \caption[The Context Diagram illustrates the modules, business processes, ...etc that feed 
or interact with this proposed application.] {The Context Diagram illustrates the modules, business processes, ...etc that feed 
or interact with this proposed application.}
\end{figure}


%==============================================================================
\newpage
\subsubsection{High Level Workflow}
The workflow required to complete the primary objectives of the proposed 
application is described below. The workflow is business-centered, and 
includes ``decision forks'' for decisions the business user, or application, 
must make to achieve the objective. The workflow omits application faults or exceptions. 
Individual tasks in the workflow are described.

\begin{my_enumerate}
  \item Workflow/Process Map 
	
\begin{figure}[H]
  \centering
\includegraphics[width=13cm]{workflow_process_map.png}
  \caption[Workflow Process] {Workflow Process}
\end{figure}

  \item Workflow Description
  \begin{my_itemize}
    \item Incoming Emails are preprocessed.
    \item Classification module uses the user model to classify the email into a specific label.
    \item The result is returned by the web service.
  \end{my_itemize}
\end{my_enumerate}

%==============================================================================
\subsubsection{User stories}
Breaking up the system components to user stories with estimated time (in terms
of man-days = 4 hours) to finish the story is very good practice.
\\

\begin{tabular}{|p{3cm}|p{10cm}|}
\hline
\cellcolor[gray]{0.9} Story Id & \#1 \\ \hline
\cellcolor[gray]{0.9} User Story & Authentication \\ \hline
\cellcolor[gray]{0.9} Priority & Low \\ \hline
\cellcolor[gray]{0.9} Description & 
      As a \textbf{User}, I can \textbf{authenticate the application}. \\ \hline
\cellcolor[gray]{0.9} Estimated time & 1 man-day \\ \hline
\cellcolor[gray]{0.9} Notes & 
      Authentication is done with email username and password. \\ \hline
\end{tabular}

\begin{tabular}{|p{3cm}|p{10cm}|}
\hline
\cellcolor[gray]{0.9} Story Id & \#2 \\ \hline
\cellcolor[gray]{0.9} User Story & Deauthentication \\ \hline
\cellcolor[gray]{0.9} Priority & Low \\ \hline
\cellcolor[gray]{0.9} Description & 
	As a \textbf{User}, I can \textbf{deauthenticate the application}. \\ \hline
\cellcolor[gray]{0.9} Estimated time & 1 man-day \\ \hline
\cellcolor[gray]{0.9} Notes & 
	Authentication is done with email username and password. \\ \hline
\end{tabular}

\begin{tabular}{|p{3cm}|p{10cm}|}
\hline
\cellcolor[gray]{0.9} Story Id & \#3 \\ \hline
\cellcolor[gray]{0.9} User Story & Label suggestion \\ \hline
\cellcolor[gray]{0.9} Priority & Medium\\ \hline
\cellcolor[gray]{0.9} Description & 
	As a \textbf{User}, I can \textbf{give feedback for the chosen/suggested labels} to
	\textbf{enhance the classification accuracy}. \\ \hline
\cellcolor[gray]{0.9} Estimated time & 5 man-days\\ \hline
\cellcolor[gray]{0.9} Notes & 
	Build browser extension to support suggestions in gmail. \\ \hline
\end{tabular}

\begin{tabular}{|p{3cm}|p{10cm}|}
\hline
\cellcolor[gray]{0.9} Story Id & \#4 \\ \hline
\cellcolor[gray]{0.9} User Story & Data pre-processing \\ \hline
\cellcolor[gray]{0.9} Priority & High\\ \hline
\cellcolor[gray]{0.9} Description & 
	As a \textbf{System}, I can \textbf{preprocess the dataset} to
	\textbf{make it ready for the classification/summarization process}. \\ \hline
\cellcolor[gray]{0.9} Estimated time & 4 man-days\\ \hline
\cellcolor[gray]{0.9} Notes & 
	Preprocessing includes stemming, checking email length/language,
	removing stop words and identifying the features. \\ \hline
\end{tabular}

\begin{tabular}{|p{3cm}|p{10cm}|}
\hline
\cellcolor[gray]{0.9} Story Id & \#5 \\ \hline
\cellcolor[gray]{0.9} User Story & Email Classification \\ \hline
\cellcolor[gray]{0.9} Priority & High\\ \hline
\cellcolor[gray]{0.9} Description & 
	As a \textbf{System}, I can \textbf{make online classification to
	an incoming email}. \\ \hline
\cellcolor[gray]{0.9} Estimated time & 20 man-days\\ \hline
\cellcolor[gray]{0.9} Notes & 
	More than one algorithm will be implemented to choose the one with 
	the best accuracy. \\ \hline
\end{tabular}

\begin{tabular}{|p{3cm}|p{10cm}|}
\hline
\cellcolor[gray]{0.9} Story Id & \#6 \\ \hline
\cellcolor[gray]{0.9} User Story & User Classification Model \\ \hline
\cellcolor[gray]{0.9} Priority & High\\ \hline
\cellcolor[gray]{0.9} Description & 
	As a \textbf{System}, I can \textbf{build a user classification 
	model from the training data}. \\ \hline
\cellcolor[gray]{0.9} Estimated time & 5 man-days\\ \hline
\cellcolor[gray]{0.9} Notes & 
	It will start after user authentication, and can be applied 
	from time to time to enhance the user model. \\ \hline
\end{tabular}

\begin{tabular}{|p{3cm}|p{10cm}|}
\hline
\cellcolor[gray]{0.9} Story Id & \#7 \\ \hline
\cellcolor[gray]{0.9} User Story & Classification Accuracy test \\ \hline
\cellcolor[gray]{0.9} Priority & Medium \\ \hline
\cellcolor[gray]{0.9} Description & 
	As an \textbf{Admin}, I can \textbf{test the accuracy of the 
	classification algorithm at runtime}. \\ \hline
\cellcolor[gray]{0.9} Estimated time & 2 man-days\\ \hline
\cellcolor[gray]{0.9} Notes & 
	The admin can view the accuracy of the used classification algorithm. \\ \hline
\end{tabular}

%==============================================================================
\newpage
\subsection{Business Requirements}

This section identifies, enumerates and explores the business requirements that must 
be met by the application. Business requirements include capturing the types of users, 
the basic inputs and outputs, the system's dependencies, and the tasks the system should 
accomplish. It is important to confirm that Task Requirements include all the tasks 
required to meet the business objectives.

%==============================================================================
\subsubsection{Users}
All applications have users and most have several users of different types. This 
section identifies, at a high level, the types of users of the system.

\begin{my_enumerate}
  \item Application User: requests emails classification.
  \item Admin: tunes classification algorithms and observes the results.
\end{my_enumerate}

%==============================================================================
\subsubsection{Inputs/Outputs}
This section identifies and describes the inputs to and outputs from the new 
application. Inputs can include electronic inputs, like RSS and EDI feeds, 
updates from external databases, ...etc, as well as human inputs, like 
``user X keys in results from report Y.'' Output can include electronic feeds, 
printed reports, ...etc. In this section, all electronic inputs and outputs 
are captured. Human inputs are omitted from this section.

\begin{my_enumerate}
  \item Inputs:
  \begin{my_itemize}
    \item unclassified Emails;
    \item classified Emails (training data);
  \end{my_itemize}
  \item Outputs:
  \begin{my_itemize}
    \item classified emails;
    \item classification model.
  \end{my_itemize}
\end{my_enumerate}


%==============================================================================
\subsubsection{Dependencies}
Human inputs are also categorized as dependencies.
\begin{my_enumerate}
  \item Registration data.
  \item Request for classifying an email.
\end{my_enumerate}


%==============================================================================
\subsubsection{Security Requirements}
This section documents, at a high level, the basic security requirements of the 
application. For example, does the system require the user to login?  Should the 
user’s identity be authenticated on just this system, or against a central authority?  
Are there any special or unusual security requirements, like fingerprint scanning?

\begin{my_enumerate}
  \item The system requires every user to login with a unique identity (email).
  \item User Data (email) should be transmitted through a secure connection.
  \item Administrators shouldn’t have access to user data (email).
\end{my_enumerate}



%==============================================================================
\subsubsection{Performance Requirements}
Performance Requirements for the application are defined at a high level below. 
If there are specific requirements for specific features to perform at a quantifiable 
level, they too are listed below. These requirements will be developed in more detail when 
the Requirements Specification is written, in a later phase.

\begin{my_enumerate}
  \item Classification web service should respond within a reasonable time.
  \item Initial training phase should finish within a reasonable time.
  \item The system will achieve high uptime.
\end{my_enumerate}


%==============================================================================
\subsubsection{Data Migration}
Data Migration describes the data that needs to be moved from an older or external 
system to the new system, in order for it to operate at launch. Migration also includes 
data that must be transferred from the new system to another external system. 
Any data migration requirements are listed below.

\begin{my_enumerate}
  \item Users' classified emails for training phase as pre-classified emails are needed to build users' models.
  \item Users' feedback on the results of the classification.
\end{my_enumerate}

%==============================================================================

\section{Software Design Description (SDD)}

%------------------------------------------------------------------------------

\subsection{Purpose}
This chapter shows how the smart email software system will be 
structured to satisfy the requirements identified in the software
requirements specification section. It is a translation of 
requirements into a description of the software structure, software components, 
interfaces, and data necessary for the implementation phase.

\subsection{Scope}
Smart email will provide automated email classification for email clients. 
This chapter describes how the system will be divided into modules and 
the design details for each module.


\subsection{Decomposition description}
This project is designed using an incremental approach. There are three
primary  stages to the design development which consists of phase 1 
(Classification Web Service Module), phase 2 (Email Monitoring Web Service Module),
and phase 3 (Web Browser Extension).

The classification module can be divided into three layers. The data layer will be 
responsible for retrieving emails from file systems or using IMAP protocol. The 
second layer is the preprocessing and email filtering layer. This layer is 
responsible for filtering emails and extracting classification features from the 
email. The third and last layer will be responsible for classifying emails and 
testing the classifier performance.

The detailed design for each design entity is illustrated in the detailed class 
diagram at the end of this section.

\subsubsection{System decomposition}
The system is divided into main modules as follows:
\begin{my_itemize}
  \item classification web service module;
  \item email monitoring web service module;
  \item web browser extension.
\end{my_itemize}
Each of these modules will be described in details in the upcoming subsections.

\paragraph{Classification Web Service Module}

\begin{my_itemize}
  \item ClassificationManager class
  \begin{my_desc}
    \item[Purpose] Used to manage and control the classification process flow
      with the dataset and classification algorithms.
% TODO
%     \item[Functions] TBD
  \end{my_desc}
  \item PreprocessingManager class
  \begin{my_desc}
    \item[Purpose] Used to manage preprocessing phase by applying selected data preprocessors
      to given raw emails.
    \item[Functions] \hfill
    \begin{my_itemize}
      \item apply(email:Email) used to apply the selected preprocessors on a given email.
    \end{my_itemize}
  \end{my_desc}

  \item FilterManager class
  \begin{my_desc}
    \item[Purpose] Used to manage feature extraction phase from preprocessed emails by applying
      specified set of filters.
    \item[Functions] \hfill
    \begin{my_itemize}
      \item getInstances(emails:List$\langle$Email$\rangle$) given a set of emails returns a prepared dataset (Instances object)
      \item getInstance(email:Email) given an email return a prepared feature vector (Instance object)
    \end{my_itemize}

  \end{my_desc}

  \item FilterCreatorManager class
  \begin{my_desc}
    \item[Purpose] Used to manage all Filter creators to create specified filters given a list of emails
    \item[Functions] \hfill
    \begin{my_itemize}
      \item getFilters(emails:List$\langle$Email$\rangle$) create all supported filters using given emails
    \end{my_itemize}

  \end{my_desc}


  \item Instances class
  \begin{my_desc}
    \item[Purpose] Used to gather all instances of the emails and treated as the classifier
      dataset.
    \item[Fuctions] \hfill
    \begin{my_itemize}
      \item add(instance:Instance) add new instance to the dataset.
      \item getAttribute(index:int) get Attribute at given index.
      \item getAttribute(name:String) get Attribute with the given name.
      \item checkInstance(instance:Instance) check if the given instance is in the dataset or not.
      \item getClassAttribute() return the class attribute of the dataset.
    \end{my_itemize}

  \end{my_desc}

  \item Instance class
  \begin{my_desc}
    \item[Purpose] Used to encapsulate the feature vector of a certain email.
    \item[Functions] \hfill
    \begin{my_itemize}
      \item getClassIndex() returns the index of the class attribute
      \item setClassIndex(index:int) set the index of the class attribute
      \item isMissingClass() check if the feature vector contains value of the class attribute or not
      \item getValue(index:int) return the value for the ith attribute
    \end{my_itemize}

  \end{my_desc}

  \item Attribute class
  \begin{my_desc}
    \item[Purpose] Used to encapsulate the dataset attribute with its type
    \item[Functions] \hfill
    \begin{my_itemize}
      \item Attribute(name:String) \\
      used to create a new Attribute instance with the given name.
      \item Attribute(name:String,nominals:ArrayList$\langle$String$\rangle$) \\
      used to create a new Attribute instance with the given name and possible values.
      \item getValue(int:index): String \\
      returns the attribute possible value of the given index
      \item getIndexofValue(value:String): int \\
      returns the index of the given attribute possible value
    \end{my_itemize}

  \end{my_desc}

\end{my_itemize}

\paragraph{Email Monitoring Web Service Module}

Email Monitoring Web Service is a web service implemented in RubyOnRails \cite{ROR} framework which integrates 
with a classification web service for classification of emails. The web service monitors 
users emails and sends every new email to the classification service through a REST API \cite{REST}
which returns a response with the email label, the web service then applies the label to the user’s email.

\begin{itemize}
 \item Features
 \begin{itemize}
    \item Users registration.
    \item Adding more than one email account.
    \item Continuous monitoring of emails accounts for new emails.
 \end{itemize}
 \item Future Work
  \begin{itemize}
    \item Providing statistics about the classification process.
    \item Support email providers other than Gmail: Yahoo, Hotmail, ...etc.
    \item Classifying users into active and non-active ones based on their 
    email account activity to decrease the load and increase the performance 
    of the monitoring process.
    \item Improving performance and security.
  \end{itemize}
\end{itemize}

\paragraph{Web Browser Extension}
Web Browser Extension is a client side browser extension that injects a 
``Classify Me'' button when the user open the email which is an additional 
UI component injected into the user's email web interface. 


\begin{itemize}
 \item Features
 \begin{itemize}
    \item Users registration.
    \item Sending a classification requests to the classification web service.
    \item Check the status of the user's training phase at the classification 
    web service.
    \item Sending feedback to the classification web service when the user
    applies labels manually to an email to re-train the model.
 \end{itemize}
 \item Future Work
  \begin{itemize}
    \item Providing statistics about the classification process.
    \item Support other web browsers other than Chrome extension: Firefox, 
    Safari, ...etc.
    \item Improving performance and security.
  \end{itemize}
\end{itemize}

\subsubsection{Data decomposition}
The data layer is responsible for retrieving emails either from the file system or using 
the IMAP protocol. The Enron dataset \cite{ENRON} will be used as the data source for training and 
testing the email classifier.

\begin{my_desc}
  \item[Data access object] this class is used to provide an abstract way to retrieve emails.
  \item[IMAP Data access object] it is responsible for retrieving emails using the IMAP protocol,
  and it is inherited from the Data access object class.
  \item[File system data access object] it is responsible for retrieving emails from the file
  system, and it is inherited from the Data access object class.
\end{my_desc}


%==============================================================================
\subsection{Interface description}
The interface description provides everything designers, programmers and 
testers need to know to correctly use the functions provided by the system 
entities. This description includes the details of external and internal 
interfaces not provided in the software requirements specification.

\subsubsection{Module interface}
\paragraph{Classification module interface}
\begin{my_itemize}
  \item Classifier class
  \begin{my_desc}
   \item[Purpose] All schemes for documents classification extend this class. Note that a classifier MUST either implement distributionForInstance() or classifyInstance() methods.
   \item[Function] The Classifier abstarct class defines the following functions:
	\begin{my_itemize}
		\item classifyInstance(Instance instance) : classifies a given instance;
		\item buildClassifier(Instances trainingData) : builds the classification model from the given set of training data;
		\item getClassifierByName(String name, String[] options) : returns an instance of the classifier given the classifier name;
		\item distributionForInstance(Instance instance): returns the destribution for each class attribute for the given instance.
	\end{my_itemize}
  \end{my_desc}

  \item Preprocessor interface
  \begin{my_desc}
   \item[Purpose] Performs some pre-processing actions on the email before classifiaction. Pre-processing may include removing stop words and stemming. 
   \item[Function] The Preprocessor interface defines the following function:
	\begin{my_itemize}
	\item process(Email email): Performs some pre-processing action on the given email such as stemming or removing stop words.
	\end{my_itemize}
  \end{my_desc}

  \item Filter class
  \begin{my_desc}
   \item[Purpose] Defines an abstract class for filtering an email. A filter is used to extract a set of features from the email such as the email sender, email label and word frequencies in an email.
   \item[Function] The filter abstract class defines the following set of functions:
	\begin{my_itemize}
	\item makeFeatureInstance(Email email): creates a feature instance from the given email.
	\item getAttributes(): returns the set of attributes for the given filter.
	\end{my_itemize}
  \end{my_desc}

  \item FilterCreator interface
  \begin{my_desc}
   \item[Purpose] Defines an interface for creating a filter.
   \item[Function] The FilterCreator interface defines the following function:
	\begin{my_itemize}
	\item createFilter(emails:List$\langle$Email$\rangle$): creates a filter given the list of emails.
	\end{my_itemize}
  \end{my_desc}

\end{my_itemize}

%-------
\newpage
\paragraph{Email Monitoring Web Service Module interface}
.\\
\begin{figure}[H]
  \centering
  \includegraphics[width=12cm]{account_example.png}
  \caption[Example for MVC in action handling a user request to view his 
  accounts]{Example for MVC in action handling a user request to view his accounts}
\end{figure}

%==============================================================================
\newpage
\subsubsection{Detailed design}
\paragraph{Classification module detailed design}
.\\
\begin{figure}[H]
  \centering
  \includegraphics[width=12cm]{design.jpeg}
  \caption[Detailed Design] {Detailed Design}
\end{figure}



\paragraph{Email Monitoring Web service module detailed design}
.\\
\begin{figure}[H]
  \centering
  \includegraphics[width=15cm]{mvc2.png}
  \caption[Smart Email MVC Design]{Smart Email MVC Design}
\end{figure}


\section{Conclusion}
In this chapter, the software requirements specification and software design
description were discussed. While in the next chapter, Environment and 
Development tools will be discussed.

% Chapter 5

\chapter{Development Process, Environment and Tools} % Main chapter title

\label{Chapter5} % For referencing the chapter elsewhere, use \ref{Chapter5} 

\lhead{Chapter 5. \emph{Development Process, Environment and Tools}} % This is for the header on each page - perhaps a shortened title

\section{Introduction}
In Chapter \ref{Chapter4}, Smart Email architecture and design was discussed. In this chapter, 
the adopted process is examined in detail; the stages of work, the tools used 
and the work environment.

The environments needed to be set up are discussed in section \ref{sec:5_setting_up_env}.
In section \ref{sec:5_sw_eng_process_followed}, 
the software engineering model followed throughout the project will be explained. 
In section \ref{sec:5_dev_process}, the development process will be described in detail. In section \ref{sec:5_tools_used}, 
all the tools used will be explained. The chapter is concluded in section \ref{sec:conclusion_5}.
%----------------------------------------------------------------------------------------
\section{Setting Up The Environment}
\label{sec:5_setting_up_env}
The needed environment for the project to run
\begin{itemize}
  \item Any operating system that is compatible with Java Runtime Environment.
  \item Java runtime environment.
  \item Tomcat 6 server.
  \item MySQL and PostgreSQL database.
  \item RubyOnRails Framework \cite{ROR}.
\end{itemize}

%--------------------------------------------------------
\section{Software Engineering Process Followed}
\label{sec:5_sw_eng_process_followed}
Since the project included many different processes and stages. A long time is 
spent and multiple team members were working on it. It was critical to follow 
a standard methodological software engineering model to keep the work organized 
and fully utilize the effort.

Software Process Improvement (SPI) for Small Medium Enterprises (SMEs)
model is a model developed by software Engineering Competence Center
whose goal is to help small and medium enterprise to raise the quality of
their products by using modern software development processes and practices. 
It consists of several processes: the management, development, peer-review, 
quality assurance and configuration management processes. Since
the SPI model targets mainly enterprise and business, it was required to
slightly modify it to fit scientific natural of the project. Three of these 
processes were followed in the work which are the development, peer review and
configuration management processes.

\begin{itemize}
  \item Development Process: the development processes contained several
  phases so it will be discusses in next section \ref{sec:5_dev_process} .
  \item Peer review Process: peer review allows to detect the errors early and
  almost all team members share their thoughts in the same point and
  look at it from many points of view. Throughout the project, all work
  was reviewed among the team members.
  \item Configuration Management Process: concerned with controlling and
  reporting changes during the project's lifetime. During the implementation 
  phase, the source code was edited frequently and by multiple
  team members. So configuration management was applied by using
  source control system.
\end{itemize}

%--------------------------------------------------------
\section{Development Process}
\label{sec:5_dev_process}

The development process adopted through the project included the following
phases: requirements specification, planning.

\subsection{Requirements Specification}
Extracting the requirements is considered one of the most important phases
in any software development. The main requirement was a classification web 
service with a REST API \cite{REST}.
\subsection{Planning}
To achieve the main requirement, several sub-requirements were planned.
First, it was important to compare between classification algorithms and choose 
a subset of them to use, then choosing the best Email feature combination to 
give the highest accuracy, then choosing the API syntax for the classification 
web service. Finally, building a web monitoring web service and a browser extension 
to demonstrate the capabilities of the classification web service.
\subsection{Analysis}
As stated in the planning phase, Classification algorithms were analyzed and 
several Email features were compared, the results of this comparisons are 
discussed in chapter 6. Enron dataset was used to do the analysis. The largest 7 
users were chosen to do the analysis \cite{RON04}.

\subsection{Design}
Since several algorithms and features combinations are analyzed and compared, 
a flexible design that enables changing the classification algorithm or the 
classification feature were needed. Design is described in detail in the previous 
chapter.

\subsection{Implementation}
The implementation phase was the most time demanding phase in the project. The 
implementation phase was divided into 3 parts. First implementing the 
classification web service using JavaEE and REST API \cite{REST},
then implementing the email monitoring web service using ruby on rails \cite{ROR}
and finally the browser extension using Google chrome browser extension API \cite{CHROME}.

\subsection{Integration}
After implementing the classification web service,
it was required to integrate the email monitoring web service and the browser extension with
the classification web service and make sure they can communicate with the classification web
service using the REST API.

\subsection{Testing}
Several levels and types of testing were applied throughout the project.
\begin{itemize}
  \item Unit testing was applied on several standalone modules.
  \item Integration testing was applied to the classification service as a whole.
\end{itemize}

\subsection{Deployment}
The classification web service was configured so that it can be deployed on any 
web server that supports javaEE. The email monitoring web service was deployed 
to heroku \cite{HEROKU}. The chrome browser extension can be deployed to chrome 
extension web store.

%--------------------------------------------------------
\section{Tools Used}
\label{sec:5_tools_used}

\subsection{Programming Languages Used}
\begin{itemize}
  \item Java Enterprise Edition version 6 was used as the main language in the 
  classification web service.
  \item Ruby on rails \cite{ROR} was used for developing the email monitoring 
  web service.
  \item Javascript was used for developing the browser extension.
\end{itemize}

\subsection{Integrated Development Environments (IDEs)}
Eclipse IDE was used for the classification web server.
Vim text editor was used for developing the browser extension and the email 
monitoring web service.

\subsection{External Libraries}
The following libraries were used
\begin{itemize}
  \item WEKA \cite{WEKA} data-mining library is used to provide implementation for SVM algorithm.
  \item Jersy \cite{JERSY} is used for building restful web application in java.
  \item Gmailr \cite{GMAILR} is used to detect events within the Gmail web interface.
\end{itemize}


\subsection{Web Servers}
The following web servers were used
\begin{itemize}
  \item Tomcat 6 is used as a java web server for the classification web server.
  \item Rails web server as a server for the email monitoring web server.
\end{itemize}

\subsection{Other Tools}
The following libraries were used
\begin{itemize}
  \item Dia Diagram : is used to draw UML and other diagrams.
  \item Matplotlib python graph module is used for drawing graphs and charts in Chapter 6.
\end{itemize}

\subsection{Source Control}
git source control system \cite{GIT} was chosen to allow versioning and provide 
the ability to rollback to any version of the project.
%--------------------------------------------------------
\section{Conclusion}
\label{sec:conclusion_5}
In this chapter, the development process, used tools and project environment were 
discussed in depth. In the next chapter, an analysis for the classification 
procedure is provided.

% Chapter 6

\chapter{Feature Analysis} % Main chapter title

\label{Chapter6} % For referencing the chapter elsewhere, use \ref{Chapter6} 

\lhead{Chapter 6. \emph{Feature Analysis}} % This is for the header on each page - perhaps a shortened title

%----------------------------------------------------------------------------------------

To better understand the individual contributions of each feature used for email classification to the overall effectiveness of the classifier, we next present an analysis of the ability of each to capture information and its effect on the overall accuracy of the classifier.

\section{Comparison Between Naive Bayes and SVM Classifiers}
\includegraphics[width=15cm]{Quality.png}

\section{Tuning the Word Frequency Feature Threshold for Naive Bayes Using Boolean Attributes}
\includegraphics[width=15cm]{WF_Threshold_Value_naiveBayes_bool1.png}

\section{Tuning the Word Frequency Filter Threshold for SVM Using Boolean Attributes}
\includegraphics[width=15cm]{WF_Threshold_Value_svm_bool1.png}

\section{Tuning the Word Frequency Filter Threshold for SVM Using Numeric Attributes}
\includegraphics[width=15cm]{WF_Threshold_Value_svm_numeric.png}

\section{Combinations between sender/WF/Size/Subject features Using Naive Bayes}
\includegraphics[width=15cm]{all_combs_naiveBayes.png}

\section{Combinations between sender/WF/Size/Subject features Using SVM}
\includegraphics[width=15cm]{all_combs_svm.png}


\section{Feature Ranking}
% Chapter 7

\chapter{Conclusion And Future Work} % Main chapter title

\label{Chapter7} % For referencing the chapter elsewhere, use \ref{Chapter6} 

\lhead{Chapter 7. \emph{Conclusion And Future Work}} % This is for the header on each page - perhaps a shortened title

%----------------------------------------------------------------------------------------


\section{Conclusion}
In chapter 1, the project's main idea was illustrated which
is a web service that provides automatic email categorization into
user-defined folders based on machine learning and data mining
classification techniques.

A classification web service was built with REST API \cite{REST} that receives
a classification request with the required email to be classified and returns
the most suitable category for this email.

An Email monitoring web service was built as a client for the classification
web service mentioned above. Users can register in this web service to have
their email accounts continuously monitored and their incoming emails automatically
classified to the best category.

A Google Chrome Browser \cite{CHROME} Extension was built as another example of a
classification web service client. The Extension provides google chrome users
with a graphical user interface that is integrated with gmail web interface.
Using this extension , Users can send email classification requests on demand.
The extension also detects manual classification done by the user to be sent
for the classification web service as a feedback on the classifier to learn
from the users classification pattern.
%================================================
\section{Future Work}
In the previous section, the implemented features were stated. In this section
the new features that could be added to Smart Email will be introduced.

\subsection{Future work related to classification web service}
\begin{itemize}
    \item Security issues: Emails privacy and security are considered very
    important issues, so an additional security to communication between clients and
    web service need to supported.
    \item Multi-Label email classification: Currently the classification web
    service assigns each email only one label. Support for multi-label
    classification can be added.
    \item Improving the classification accuracy.
\end{itemize}

\subsection{Future work related to other Smart Email components}
\begin{itemize}
    \item summarization of email threads can be supported.
    \item Supporting other browsers like (Firefox and Internet Explorer).
\end{itemize}




%----------------------------------------------------------------------------------------
%	THESIS CONTENT - APPENDICES
%----------------------------------------------------------------------------------------

\addtocontents{toc}{\vspace{2em}} % Add a gap in the Contents, for aesthetics

\appendix % Cue to tell LaTeX that the following 'chapters' are Appendices

% Include the appendices of the thesis as separate files from the Appendices folder
% Uncomment the lines as you write the Appendices

\input{./Appendices/AppendixA}
% Appendix Template

\chapter{Technical Background for Email Monitoring Web Service} % Main appendix title

\label{AppendixC} % Change X to a consecutive letter; for referencing this appendix elsewhere, use \ref{AppendixX}

\lhead{Appendix C. \emph{Technical Background for Email Monitoring Web Service}} % Change X to a consecutive letter; this is for the header on each page - perhaps a shortened title



\section{Ruby on Rails Web Framework}
\begin{itemize}
 \item Ruby On Rails is an open source full stack web application framework the ruby 
      programming language. Like many other frameworks, Rails uses the 
      Model/View/Controller (MVC) architecture pattern to organize application 
      programming which makes it easier to develop, maintain and deploy web applications.
  \item Smart Email application uses Rails 3.2 and Ruby 1.9.2
\end{itemize}

\section{Heroku Deployment Server}
Heroku is a cloud application platform which provides a platform as a service for 
building, deploying and running cloud applications. Deployment is done by pushing 
to a git repository for continuous deployment.

Also an addon named logentries integrated with heroku is used, this addon is a 
complete log management system for  giving alerts when serious errors occured to 
the application, logging all applications events and requests and this beside 
the normal log management features such as Tail, Search, Visualization, and Notifications.


\section{Database}
\begin{itemize}
 \item Production:
  \begin{itemize}
   \item PostgreSQL: it is a powerful open source object relational database system 
      with architecture that have proven reliability, data integrity, and correctness. 
      PostgreSQL runs on almost all major platforms including Linux, Windows and Mac OS.
  \end{itemize}
 \item Development:
 \begin{itemize}
  \item MySQL: it is the world’s most popular open source database because of it’s performance,
      high reliability and ease of use. MySQL runs on more than 20 platforms including Linux, Windows and Mac OS.
 \end{itemize}
\end{itemize}

\section{RESTful Web Service}
REST is a style of software architecture for distributed systems which uses noun and verbs to describe 
web pages. Rest describes six constraints applied to the architecture:
\begin{itemize}
 \item Client-Server: a uniform interface which separate server from client and accessed with HTTP Get, Post, Put, and Delete.
 \item Stateless.
 \item Cache-able.
 \item Layered System.
 \item Named Resources.
 \item Hyperlinks.
\end{itemize}

Also REST have some advantages over SOAP:
\begin{itemize}
 \item Simpler.
 \item Caching.
 \item Easier state transitions with hyperlinks and no intermediate stubs as in SOAP i.e data goes directly from client to server.
\end{itemize}

\section{Resque-Redis Job Scheduler}
\begin{itemize}
 \item Redis is an open source, advanced key-value store which is often refers to as a data structure server.
 \item Resque is a redis backed library for creating background jobs, placing these jobs in queues and 
      later processing them. Resque queues are persistent, provide visibility to their content and 
      store jobs as simple JSON packages.
 \item Resque is used  in Smart Email to schedule the monitor job which monitors the users emails every 30 minutes. 
\end{itemize}

\section{IMAP}
\begin{itemize}
 \item An application layer internet protocol that allows an email client to access emails on a remote mail server.
 \item IMAP is used to access users email accounts in the classification web service and application web service.
 \item IMAP-IDLE:
 \begin{itemize}
  \item IMAP feature which enables the email client connected to the mail server to be notified when a new email arrives.
  \item It is used in Smart Email to monitor users emails to be able to send the email to the classification service as soon as the email arrives.
 \end{itemize}
\end{itemize}

%\input{./Appendices/AppendixC}

\addtocontents{toc}{\vspace{2em}} % Add a gap in the Contents, for aesthetics

\backmatter

%----------------------------------------------------------------------------------------
%	BIBLIOGRAPHY
%----------------------------------------------------------------------------------------

\label{Bibliography}

\lhead{\emph{Bibliography}} % Change the page header to say "Bibliography"

\bibliographystyle{unsrt} % Use the "unsrt" BibTeX style for formatting the Bibliography

\bibliography{Bibliography} % The references (bibliography) information are stored in the file named "Bibliography.bib"

\end{document}  
