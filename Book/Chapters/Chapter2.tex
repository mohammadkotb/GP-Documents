% Chapter 2
\newenvironment{my_itemize}
{\begin{itemize}
  \setlength{\itemsep}{0cm}
  \setlength{\parskip}{0cm}}
{\end{itemize}}
\newenvironment{my_enumerate}
{\begin{enumerate}
  \setlength{\itemsep}{0cm}
  \setlength{\parskip}{0cm}}
{\end{enumerate}}

\chapter{Survey and Taxonomy for Email Classification and Summarization Research Papers} % Main chapter title

\label{Chapter2} % For referencing the chapter elsewhere, use \ref{Chapter2} 

\lhead{Chapter 2. \emph{Survey and Taxonomy}} % This is for the header on each page - perhaps a shortened title

%----------------------------------------------------------------------------------------

\section{Introduction}
Email has been an efficient and popular communication mechanism as the 
number of Internet users increases. Therefore, email management has become 
an important and growing problem for individuals and organizations because 
it is prone to misuse. One of the problems that are most important is disordered 
email message, congested and unstructured emails in mail boxes. It may be very 
hard to find archived email message, search for previous emails with specified 
contents or features when the mails are not well structured and organized.

Many machine learning approaches have been applied in this field, the most 
State-of-the-Art algorithms in email classification include: support vector 
machines, neural network, Na\"{\i}ve Bayes classifiers and entropy-based approach. 

Email summarization is another important and challenging problem. We can think 
of automatic summarization as a type of information compression. To achieve such 
compression, better modelling and understanding of document structures and internal 
relations is required. 

In this chapter we present a survey and taxonomy on recent research topics 
related to email classification and summarization.

%==============================================================================
\newpage
\section{Email Classification Taxonomy}
The following table classifies some recent research papers in the field of email classification.

\subsection{Classification according to the different learning algorithms used in different papers}

\begin{center}
\begin{tabular}{|p{2cm}|p{2cm}|p{2cm}|p{2cm}|p{2cm}|p{2cm}|}
\hline
\multicolumn{6}{|c|}{Learning Algorithm} \\
\hline
SVM & Na\"{\i}ve Bayes & Neural Networks & Max. Entropy / Winnow & Nnge / Hoeffing Trees & Graph Mining \\ \hline
Email Classification with Co-training \cite{SVETLANA01} &
Email Classification with Co-training \cite{SVETLANA01} &
Email Classification: Solution with Back Propagation Technique \cite{mous05} & 
Automatic Categorization of Emails into Folders \cite{RON04} &
Using GNUsmail to compare Data Stream Mining Methods for On-line Email Classification \cite{JOSE11} &
A graph Based Approach for Multi-Folder Email Classification \cite{sift02} \\ \hline

Automatic Categorization of Emails into Folders \cite{RON04} &
Automatic Categorization of Emails into Folders \cite{RON04} &
Email Classification Using Semantic Feature Space \cite{YUN08} & 
&
& \\ \hline
\end{tabular}
\end{center}
\newpage

\subsection{Classification according to the different learning capabilities}

\begin{center}
\begin{tabular}{|p{6cm}|p{6cm}|}
\hline
\multicolumn{2}{|c|}{Learning Capability} \\
\hline
On-line Learning & Off-Line Learning 
\\ \hline
Using GNUsmail to Compare Data Stream Mining Methods \cite{JOSE11} &
An Object Oriented Email Clustering Model Using  Weighted Similarities 
between Email Attributes \cite{NARESH10}
\\ \hline

GNUsmail: Open Framework for On-line Email Classification \cite{MANUEL11}
& Content Based Email Classification System by applying Conceptual Maps \cite{BASKARAN09}
\\ \hline

& E-Classifier: A Bi-Lingual Email Classification System \cite{NOUF08}
\\ \hline

& Email classification for contact centers \cite{ANI03}
\\ \hline

& 
Automatic Categorization of Email into Folders \cite{RON04}

\\ \hline

\end{tabular}
\end{center}

\newpage

\subsection{Datasets used in Email Classification}
\subparagraph{Enron Dataset}
    \begin{my_itemize}
        \item Automatic Categorization of Email into Folders \cite{RON04}
        \item An Object Oriented Email Clustering Model Using  Weighted Similarities 
  between Email Attributes \cite{NARESH10}
        \item Using GNUsmail to Compare Data Stream Mining Methods \cite{JOSE11}
    \end{my_itemize}
\subparagraph{SRI Dataset}
    \begin{my_itemize}
        \item Automatic Categorization of Email into Folders \cite{RON04}
    \end{my_itemize}
\subparagraph{Pine Dataset}
    \begin{my_itemize}
        \item Email classification for contact centers \cite{ANI03}
    \end{my_itemize}
\subparagraph{Private Dataset}
    \begin{my_itemize}
        \item Enterprise Email Classification Based on Social Network Features \cite{MIN11}
        \item E-Classifier: A Bi-Lingual Email Classification System \cite{NOUF08} 
        \item Automatically tagging email by leveraging other users folders \cite{YEHUDA11}
    \end{my_itemize}
\subparagraph{Public Pua}
    \begin{my_itemize}
        \item Email Categorization Using Multi-Stage Classification Technique \cite{MD07}
    \end{my_itemize}

%================================= FEATURES
\subsection{Different features used for email classification}
This section summarizes the different features used for email classification in different research papers
    \subparagraph{Automatic Categorization of Email into Folders \cite{RON04}}
	\begin{my_itemize}
		\item bag-of-words document representation: messages are represented as vectors of word counts.
		\item Words are downcased.
		\item 100 most frequent words and words that appear only once in the training set are removed, and the remaining words are counted in each message to compose a vector.
		\item In future work, richer representations will be considered, including the following:
			\begin{itemize}
				\item Different sections of each email can be treated differently. For example, the system could create distinct features for words appearing in the header, body, signature, attachments, etc.
				\item Named entities may be highly relevant features.
			\end{itemize}
	\end{my_itemize}

    \subparagraph{Email Classifications For Contact Centers \cite{ANI03}}
		\begin{my_itemize}
			\item Feature sets used for experiments included:
				\begin{itemize}
					\item Non-inflected words.
					\item Noun phrases.
					\item Verb phrases.
					\item Punctuation.
					\item Length of the Email.
					\item Dictionaries.
				\end{itemize}
		\end{my_itemize}

	\subparagraph{Using GNUsmail to Compare Data Stream Mining Methods for On-line Email \cite{JOSE11}}
		\begin{my_itemize}
			\item The main feature of the text preprocessing module is a multi-layer filter structure, responsible for performing feature extraction tasks.
			\item The Inbox and Sent folders are skipped in the learning process because they can be thought of as non-specific folders.
			\item Every mail belonging to any other folder (that is, to any topical folder ) goes through a pipeline of linguistic operators which extract relevant features from it.
			\item As the number of possible features is prohibitively large, only the most relevant ones are selected.
		\end{my_itemize}
   
	\subparagraph{Content Based Email Classification System by applying Conceptual Maps \cite{BASKARAN09}}
		\begin{my_itemize}
			\item Unstructured text: consists of fields like the subject and body.
			\item Categorical text: includes fields such as "to" and "from".
			\item Numeric data: includes such features as the message size, number of
recipients and counts of particular characters.
		\end{my_itemize}

\newpage
\subsection{Chronological sort of classification papers}
\subparagraph{2011}
\begin{my_itemize}
  \item Using GNUsmail to Compare Data Stream Mining Methods for On-line Email Classification \cite{JOSE11}
  \item Enterprise Email Classification Based on Social Network Features \cite{MIN11}
  \item Automatically tagging email by leveraging other users folders \cite{YEHUDA11}
\end{my_itemize}

\subparagraph{2010}
\begin{my_itemize}
  \item An Object Oriented Email Clustering Model Using Weighted Similarities between Emails Attributes \cite{NARESH10}
  \item A Graph-Based Approach for Multi-Folder Email Classification \cite{sift02}
\end{my_itemize}

\subparagraph{2009}
\begin{my_itemize}
  \item Content Based Email Classification System by applying Conceptual Maps \cite{BASKARAN09}
  \item Email Classification: Solution with Back Propagation Technique \cite{mous05}
\end{my_itemize}

\subparagraph{2008}
\begin{my_itemize}
  \item A new approach to Email classification using Concept Vector Space Model \cite{CHAO08}
  \item E-Classifier: A Bi-Lingual Email Classification System \cite{NOUF08}
  \item Ontology based classification and categorization of email \cite{BALAKUMAR08}
  \item Applying Machine learning Algorithms for Email Management \cite{mous03}
\end{my_itemize}

\subparagraph{2007}
\begin{my_itemize}
  \item Email Categorization Using Multi-Stage Classification Technique \cite{MD07}
\end{my_itemize}

\subparagraph{2005}
\begin{my_itemize}
  \item An Email Classification Model Based on Rough Set Theory \cite{WENQING05}
  \item eMailSift: Email Classification Based on Structure and Content \cite{sift01}
\end{my_itemize}

\subparagraph{2004}
\begin{my_itemize}
  \item Automatic Categorization of Email into Folders \cite{RON04}
  \item Co-training with a Single Natural Feature Set Applied to Email Classification \cite{mous04}
\end{my_itemize}

\subparagraph{2003}
\begin{my_itemize}
  \item Email Classifications For Contact Centers \cite{ANI03}
\end{my_itemize}


\subsection{Conclusion}
\begin{my_itemize}
    \item SVM Achieved the best results in most papers, but it is computationally expensive and has the hardest implementation.
    \item The aasic form of Na\"{\i}ve Bayes algorithm has the simplest implementation but has very low classification accuracy compared to other algorithms.
    \item Online learning techniques are still under development, they are very hard to implement but characterized by their
    ability to classify new coming email without rebuilding the model.
    \item Offline learning techniques are used in most papers.
    \item Enron dataset is the most commonly used.
\end{my_itemize}

\newpage
%==============================================================================
\section{Email Summarization Taxonomy}

\subsection{Summarization taxonomy according to different techniques}

\begin{center}
\begin{tabular}{|p{6cm}|p{6cm}|}
\hline
\multicolumn{2}{|c|}{Summarization Techniques} \\
\hline
Extractive Summarization & Question-Answer Pairs detection
\\ \hline
Detection of question-answer pairs in email conversations \cite{LOKESH04} &
Using Question-Answer Pairs in Extractive Summarization of Email Conversations \cite{KATHLEEN07} 
\\ \hline

Summarizing email conversations with clue words \cite{GIUSEPPE07} &
Using Question-Answer Pairs in Extractive Summarization of Email Conversations \cite{KATHLEEN07}
\\ \hline

\end{tabular}
\end{center}


%==============================================================================
\subsection{Chronological sort of summarization papers}

\subparagraph{2001}
\begin{itemize}
  \item Combining Linguistic and Machine Learning Techniques for Email
Summarization \cite{SMAR01}
\end{itemize}

\subparagraph{2004}
\begin{itemize}
  \item Detection of question-answer pairs in email conversations \cite{LOKESH04}
\end{itemize}

\subparagraph{2007}
\begin{itemize}
  \item Using Question-Answer Pairs in Extractive Summarization of Email Conversations \cite{KATHLEEN07}
  \item Summarizing email conversations with clue words \cite{GIUSEPPE07}
\end{itemize}

\subparagraph{2009}
\begin{itemize}
  \item Regression-Based Summarization of Email Conversations \cite{JAN09}
\end{itemize}

\section{Conclusions}\label{conclusions}
    \begin{my_itemize}
        \item We will implement two classification algorithms and compare their results: SVM and Na\"{\i}ve Bayes.
        \item We will make use of the Enron dataset in learning and training phases.
        \item We will adopt an offline classification technique at the begining and will see if we can try an online technique later on.
        \item We will make use of all the features implemented in the above papers and try combining some of them to reach the best result.o
        \item We will use Weka (Waikato Environment for Knowledge Analysis) which is a collection of machine learning algorithms for data mining tasks , not to reimplement the classifications algrithms from the beginning 

    \end{my_itemize}

%=============================================================================================
% References
\begin{thebibliography}{99}
\bibitem{RON04}
  Ron Bekkerman,
  Andrew McCallum,
  Gary Huang,
  \emph{Automatic Categorization of Email into Folders: Benchmark Experiments on Enron and SRI Corpora},
  2004.

\bibitem{ANI03}
  Ani Nenkova,
  Amit Bagga,
  \emph{Email Classification for Contact Centers},
  2003.

\bibitem{JOSE11}
  Jose M. Carmona-Cejudo,
  Manuel Baena-Garcia,
  Jose del Campo-Avila,
  Rafael Morales-Bueno,
  Joao Gama,
  Albert Bifet,
  \emph{Using GNUsmail to Compare Data Stream Mining Methods for On-line Email Classification},
  2011.

\bibitem{NOUF08}
  Nouf Al Fe'ar,
  Einas Al Turki,
  Asma Al Zaid,
  Mashael Al Duwais,
  Mona Al Sheddi,
  Nora Al khamees,
  Nouf Al Drees,
  \emph{E-Classifier: A Bi-Lingual Email Classification System},
  2008.

\bibitem{NARESH10}
  Naresh Kumar Nagwani,
  Ashok Bhansali,
  \emph{An Object Oriented Email Clustering Model Using Weighted Similarities between Emails Attributes},
  2010.


\bibitem{BASKARAN09}
  S. Baskaran,
  \emph{Content Based Email Classification System by applying Conceptual Maps},
  2009.

\bibitem{CHAO08}
  Chao Zeng,
  Zhao Lu,
  Junzhong Gu,
  \emph{A new approach to Email classification using Concept Vector Space Model},
  2009.

\bibitem{BALAKUMAR08}
  M.Balakumar,
  V.Vaidehi,
  \emph{Ontology based classification and categorization of email},
  2008.

\bibitem{MIN11}
  Min-Feng Wang,
  Sie-Long Jheng,
  Meng-Feng Tsai,
  Cheng-Hsien Tang,
  \emph{Enterprise Email Classification Based on Social Network Features},
  2011.

\bibitem{MD07}
  Md Rafiqul Islam,
  Wanlei Zhou,
  \emph{Email Categorization Using Multi-Stage Classification Technique},
  2007.

\bibitem{YEHUDA11}
  Yehuda Koren,
  Edo Liberty,
  Yoelle Maarek,
  Roman Sandler,
  \emph{Automatically Tagging Email by Leveraging Other Users' Folders},
  2011.

\bibitem{WENQING05}
  Wenqing Zhao,
  Zili Zhang,
  \emph{An Email Classification Model Based on Rough Set Theory},
  2005.

\bibitem{sift01}
  Lokesh Shrestha,
  Kathleen McKeown,
  \emph{eMailSift: Email Classification Based on Structure and Content},
  2004.

\bibitem{sift02}
  Sharma Chakravarthy,
  Aravind Venkatachalam,
  Aditya Telang,
  \emph{ A Graph-Based Approach for Multi-Folder Email Classification},
  2010.

\bibitem{mous03}
  Taiwo Ayodele,
  Shikun Zhou,
  \emph{Applying Machine learning Algorithms for Email Management},
  2008.

\bibitem{mous04}
  Jason Chan,
  Irena Koprinska,
  Josiah Poon,
  \emph{Co-training with a Single Natural Feature Set Applied to Email Classification},
  2004.

\bibitem{mous05}
  Jason Chan,
  Irena Koprinska,
  Josiah Poon,
  \emph{Email Classification: Solution with Back Propagation Technique},
  2009.

\bibitem{LOKESH04}
  Lokesh Shrestha,
  Kathleen McKeown,
  \emph{Detection of question-answer pairs in email conversations},
  2004.

\bibitem{KATHLEEN07}
  Kathleen McKeown,
  Lokesh Shrestha,
  Owen Rambow,
  \emph{Using Question-Answer Pairs in Extractive Summarization of Email Conversations},
  2007.

\bibitem{GIUSEPPE07}
  Giuseppe Carenini,
  Raymond T. Ng,
  Xiaodong Zhou,
  \emph{Summarizing Email Conversations with Clue Words},
  2007.

\bibitem{SMAR01}
  Smaranda Muresan,
  Evelyne Tzoukermann,
  Judith L. Klavans,
  \emph{Combining Linguistic and Machine Learning Techniques for Email
Summarization},
  2001.

\bibitem{JAN09}
Jan Ulrich,
Giuseppe Carenini,
Gabriel Murray,
Raymond Ng
  \emph{Regression-Based Summarization of Email Conversations},
  2009.

\bibitem{SVETLANA01}
  Svetlana Kiritchenko,
  Stan Matwin,
  \emph{Email Classification with Co-Training},
  2001.

\bibitem{YUN08}
  Yun Fei Yi,
  Cheng Hua Li,
  Wei Song,
  \emph{Email classification Using Semantic Feature Space},
  2008.

\bibitem{MANUEL11}
  Jose M. Carmona-Cejudo,
  Manuel Baena-Garcia,
  Jose del Campo-Avila,
  Rafael Morales-Bueno,
  Albert Bifet,
  \emph{GNUsmail: Open Framework for On-line Email Classification},
  2011.
\end{thebibliography}
