


% Chapter 2
\newenvironment{my_itemize}
{\begin{itemize}
  \setlength{\itemsep}{0cm}
  \setlength{\parskip}{0cm}}
{\end{itemize}}
\newenvironment{my_enumerate}
{\begin{enumerate}
  \setlength{\itemsep}{0cm}
  \setlength{\parskip}{0cm}}
{\end{enumerate}}

\chapter{Background and Related Work} % Main chapter title

\label{Chapter2} % For referencing the chapter elsewhere, use \ref{Chapter2} 

\lhead{Chapter 2. \emph{Background and Related Work}} % This is for the header on each page - perhaps a shortened title

%----------------------------------------------------------------------------------------

\section{Introduction}
In chapter \ref{Chapter1}, a general description of the problem was presented, the motivation 
to develop Smart Email, and the scope of work were introduced.

In this chapter, a background about text and email classification is introduced in
sections \ref{sec:ml_text_categ}, \ref{sec:app_email_categ}, 
\ref{sec:ml_approach_text_classification}, and \ref{sec:construction_text_classifier} 
successively. In section \ref{sec:app_email_categ}, applications of email 
categorization are presented. The machine learning approach for text 
classification is discussed in section \ref{sec:ml_approach_text_classification}. 
In section \ref{sec:construction_text_classifier}, construction of text classifier 
and different types of classifiers are presented. Section \ref{sec:related_work} includes the summary of surveying several papers related to email classification. Then, a comparison
between related applications will be outlined in section \ref{sec:related_apps}. Summarizing the need to extend related applications is explained in section
\ref{sec:need_to_extend}. The chapter is concluded in section \ref{sec:conclusion_2}.

%==============================================================================

\section{Machine Learning in Automated Text Categorization}
\label{sec:ml_text_categ}

\subsection{Definition of Text Categorization}
Text categorization is the task of assigning a Boolean value to each pair 
$\langle$ d$_{j}$,c$_{i}$ $\rangle$ $\in$ D X C,  where D is a domain of documents 
and C = \{c$_{1}$, . . . , c $_{|C|}$\} is a set of predefined categories. A value 
of T assigned to $\langle$ d$_{j}$,c$_{i}$ $\rangle$ indicates a decision to 
file d$_{j}$ under c$_{i}$, while a value of F indicates a decision not to 
file d$_{j}$ under c$_{i}$. More formally, the task is to approximate the unknown 
target function $\theta$ : D ×C $\rightarrow$ \{T, F\} (that describes how documents 
ought to be classified) by means of a function $\phi$ : D × C $\rightarrow$ \{T, F\} 
called the classifier (aka rule, or hypothesis, or model ) such that $\theta$ and 
$\phi$ coincide as much as possible.\cite{Sebastiani2002}

\subsection{Single-label vs Multi-label Text Categorization}
Classification is a very important topic within supervised learning field. 
Although the most popular task for classification usually deals with 
single-label datasets, where every example is associated with a single label 
$\lambda$ from a set of disjoint labels L, the multi-label datasets are emerging 
and gaining interest due to their increasing application to real problems. Multi-label 
datasets are used when the examples are associated with a set of labels Y 
$\subseteq$ L, as occurs with email classification, image annotation, or genomics.

Two main tasks can be defined when learning from multi-label data: 
\begin{itemize}
  \item multi-label classification (MLC) that returns a subset of labels to be 
  associated with a given example (it can be considered as a bipartition of the 
  label set considering relevant and irrelevant elements);
  \item label ranking (LR) that returns an ordering of the labels according to 
  their relation with the example. \cite{Carmona2011}
\end{itemize}

\subsection{Category Pivoted vs Document Pivoted Text Categorization}
There are two different ways of using a text classifier. Given d$_{j}$ $\in$ D, 
we might want to find all the c$_{i}$ $\in$ C under which it should be filed 
(document-pivoted categorization – DPC); alternatively, given c$_{i}$ $\in$ C, 
we might want to find all the d$_{j}$ $\in$ D that should be filed under it 
(category-pivoted categorization – CPC). DPC is thus suitable when documents 
become available at different moments in time, e.g. in filtering e-mail. CPC 
is instead suitable when: 
\begin{itemize}
  \item a new category c$_{|C|+1}$ may be added to an existing set 
  C = \{c$_{1}$, . . . , c$_{|C|}$\} after a number of documents have 
  already been classified under C;
  \item these documents need to be reconsidered for classification under c$_{|C|+1}$.
\end{itemize}
DPC is used more often than CPC, as the former situation is more common than 
the latter. \cite{Sebastiani2002}

\section{Applications of Email Categorization}
\label{sec:app_email_categ}

\subsection{Spam Detection}
Spam remains a serious problem today because it continues to be a very 
profitable business for spammers. Spam takes on various forms from adult 
content, selling products/services, pharmaceuticals to stock promotions, 
job offers, etc.

Applying machine learning techniques for spam email detections plays a major 
role in protecting end users from spam email messages.\cite{peifeng2007}

\subsection{Email Organization}
Email has been an efficient and popular communication mechanism as the 
number of Internet users’ increases. Therefore, email management has 
become an important and growing problem for individuals and organizations 
because it is prone to misuse. One of the problems that are most paramount 
is disordered email message, congested and unstructured emails in mail 
boxes.

It may be very hard to find archived email message, search for previous 
emails with specified contents or features when the mails are not well 
structured and organized.

At this stage new effective method for managing information in email, 
reducing email overloads is developed by classifying emails based on 
important words. \cite{taiwo2007}

\section{The Machine Learning Approach for Text Classification}
\label{sec:ml_approach_text_classification}

In the 80s the most popular approach (at least in operational settings) for the
creation of automatic document classifiers consisted in manually building, by means
of knowledge engineering (KE) techniques, an expert system capable of taking
Text Categorization (TC) decisions. Such an expert system would typically
consist of a set of manually defined logical rules.

Since the early 90s, the Machine Learning (ML) approach to TC has gained 
popularity and has eventually become the dominant one. In ML terminology, the 
classification problem is an activity of supervised learning, since the learning 
process is `supervised' by the knowledge of the categories and of the training 
instances that belong to them.

The advantages of the ML approach over the KE approach are evident. The 
engineering effort goes towards the construction not of a classifier, 
but of an automatic builder of classifiers (the learner). \cite{Sebastiani2002}

\subsection{Training Set, Test Set, and Validation Set}
The ML approach relies on the availability of an initial 
corpus = \{d$_{1}$, . . . , d$_{|\Omega|}$\} $\subset$ D of documents 
pre-classified under C = \{c$_{1}$, . . . , c$_{|C|}$\}. That is, the values of 
the total function $\theta$ : D X C $\rightarrow$ \{T, F\} are known for every 
pair $\langle$ d$_{j}$ , c$_{i}$ $\rangle$ $\in$   $\Omega$ X C. A document 
d$_{j}$ is a positive example of c$_{i}$ if (d$_{j}$ , c$_{i}$) = T , a negative 
example of c$_{i}$ if (d$_{j}$ , c$_{i}$) = F.

The following subsection represents the different datasets used in different research papers for email classification.
\subsubsection{Datasets used in Email Classification}
\subparagraph{Enron Dataset}
    \begin{my_itemize}
        \item Automatic Categorization of Email into Folders \cite{RON04}
        \item An Object Oriented Email Clustering Model Using  Weighted Similarities 
  between Email Attributes \cite{NARESH10}
        \item Using GNUsmail to Compare Data Stream Mining Methods \cite{JOSE11}
    \end{my_itemize}
\subparagraph{SRI Dataset}
    \begin{my_itemize}
        \item Automatic Categorization of Email into Folders \cite{RON04}
    \end{my_itemize}
\subparagraph{Pine Dataset}
    \begin{my_itemize}
        \item Email classification for contact centers \cite{ANI03}
    \end{my_itemize}
\subparagraph{Private Dataset}
    \begin{my_itemize}
        \item Enterprise Email Classification Based on Social Network Features \cite{MIN11}
        \item E-Classifier: A Bi-Lingual Email Classification System \cite{NOUF08} 
        \item Automatically tagging email by leveraging other users folders \cite{YEHUDA11}
    \end{my_itemize}
\subparagraph{Public Pua}
    \begin{my_itemize}
        \item Email Categorization Using Multi-Stage Classification Technique \cite{MD07}
    \end{my_itemize}

\section{Construction of Text Classifier}
\label{sec:construction_text_classifier}
The construction of text classifiers has been tackled in a variety of ways.
In this section we will deal only with the methods that have been most popular 
in text classification. We start by discussing the general form that a text 
classifier has.  In subsections 2.5.1 to 2.5.3 we discuss number of approaches 
that have been applied in the text classification literature. In general we 
will assume the presentation of the algorithms will focus on the methods for 
classifier learning rather than on the effectiveness and efficiency of the 
classifiers built by means of them.\cite{Sebastiani2002}

\subsection{Probabilistic classifiers}
The construction of a ranking classifier for category c$_{i}$ $\in$ C usually
consists in the definition of a function CSV$_{i}$ : D $\rightarrow$ [0, 1] 
that, given a document d$_{j}$ , returns a categorization status value for it, 
i.e. a number between 0 and 1 that, roughly speaking, represents the evidence 
for the fact that d$_{j}$ $\in$ c$_{i}$.

Probabilistic classifiers view CSV$_{i}$(d$_{j}$) in terms of $P(c_{i} | d_{j})$, 
i.e. the probability that a document d$_{j}$ belongs to c$_{i}$, and compute this 
probability by an application of Bayes’ theorem, given by 
\[ P(c_{i}|d_{j}) = \frac{P(c_{i})P(d_{j}|c_{i})}{P(d_{j})} \]

In order to alleviate this problem it is common to make the assumption that 
any two coordinates of the document vector are, when viewed as random variables, 
statistically independent of each other; this independence assumption is 
encoded by the equation
\[ P(d_{j}|c_{i}) = \prod_{k=1}^{|\tau|} P(w_{kj}|c_{i}) \]

Probabilistic classifiers that use this assumption are called Naive Bayes classifiers, 
and account for most of the probabilistic approaches to TC in the literature.

\subsection{Building classifiers by support vector machines}
The support vector machine (SVM) method has been introduced in TC by Joachims 
[1998, 1999] and subsequently used in [Drucker et al. 1999; Dumais et al. 1998; 
Du- mais and Chen 2000; Klinkenberg and Joachims 2000; Taira and Haruno 1999;

Yang and Liu 1999]. In geometrical terms, it may be seen as the attempt to
find, among all the surfaces $\sigma 1$,$ \sigma 2$, ... in $|T|$ -dimensional
space that separate the positive from the negative training examples (decision
surfaces), the $\sigma i$ that separates the positives from the negatives by the 
widest possible margin, i.e. such that the separation property is invariant with 
respect to the widest possible translation of $\sigma i$ .

This idea is best understood in the case in which the positives and the
negatives are linearly separable, in which case the decision surfaces are
($|T|$-1)-hyperplanes.

In the 2-dimensional case , various lines may be chosen as decision surfaces. 
The SVM method chooses the middle element from the ``widest'' set of parallel 
lines, i.e. from the set in which the maximum distance between two elements in 
the set is highest. It is noteworthy that this ``best'' decision surface is 
determined by only a small set of training examples, called the support vectors.

\subsection{On-line Methods}
Methods for learning linear classifiers are often partitioned in two broad 
classes, batch methods and on-line methods.

Batch methods build a classifier by analyzing the training set all at once. 
On-line (aka incremental) methods build a classifier soon after examining 
the first training document, and incrementally refine it as they examine new ones. 
This may be an advantage in the applications in which the `meaning' of the 
category may change in time, as e.g. in adaptive filtering. This is also suitable 
for applications (e.g. semi-automated classification, adaptive filtering) in 
which we may expect the user of a classifier to provide feedback on how test 
documents have been classified, as in this case further training may be performed 
during the operating phase by exploiting user feedback.

\newpage
\section{Related Work}
This section contains the summary of surveying several papers related to automatic email categorization into folders.

\label{sec:related_work}

\subsection{Taxonomy of research papers according to the different learning algorithms used in different papers}
The following table classifies some recent research papers in the field of email 
classification according to the different learning algorithms for email classification.

\begin{center}
  \begin{table}[H]
    \begin{tabular}{|p{2cm}|p{2cm}|p{2cm}|p{2cm}|p{2cm}|p{2cm}|}
      \hline
      \multicolumn{6}{|c|}{Learning Algorithm} \\
      \hline
      SVM & Na\"{\i}ve Bayes & Neural Networks & Max. Entropy / Winnow & Nnge / Hoeffing Trees & Graph Mining \\ \hline
      Email Classification with Co-training \cite{SVETLANA01} &
      Email Classification with Co-training \cite{SVETLANA01} &
      Email Classification: Solution with Back Propagation Technique \cite{mous05} & 
      Automatic Categorization of Emails into Folders \cite{RON04} &
      Using GNUsmail to compare Data Stream Mining Methods for On-line Email Classification \cite{JOSE11} &
      A graph Based Approach for Multi-Folder Email Classification \cite{sift02} \\ \hline

      Automatic Categorization of Emails into Folders \cite{RON04} &
      Automatic Categorization of Emails into Folders \cite{RON04} &
      Email Classification Using Semantic Feature Space \cite{YUN08} & 
      &
      & \\ \hline
    \end{tabular}
    \caption{Taxonomy of research papers according to the different learning algorithms used in different papers.}
  \end{table}
\end{center}

\subsection{Taxonomy of research papers according to the different learning capabilities}
The following table classifies some recent research papers in the field of email 
classification according to the different learning capabilities (on-line or off-line) 
for email classification.

\begin{center}
  \begin{table}[H]
    \begin{tabular}{|p{6cm}|p{6cm}|}
      \hline
      \multicolumn{2}{|c|}{Learning Capability} \\
      \hline
      On-line Learning & Off-Line Learning 
      \\ \hline
      Using GNUsmail to Compare Data Stream Mining Methods \cite{JOSE11} &
      An Object Oriented Email Clustering Model Using  Weighted Similarities 
      between Email Attributes \cite{NARESH10}
      \\ \hline
      GNUsmail: Open Framework for On-line Email Classification \cite{MANUEL11}
      & Content Based Email Classification System by applying Conceptual Maps \cite{BASKARAN09}
      \\ \hline
      & E-Classifier: A Bi-Lingual Email Classification System \cite{NOUF08}
      \\ \hline
      & Email classification for contact centers \cite{ANI03}
      \\ \hline
      & 
      Automatic Categorization of Email into Folders \cite{RON04}
      \\ \hline
    \end{tabular}
    \caption{Taxonomy of research papers in the field of email classification 
    according to the different learning capabilities .}
  \end{table}
\end{center}



%================================= FEATURES===============================
\subsection{Different features used for email classification}
This section summarizes the different features used for email classification 
in different research papers
\subparagraph{Automatic Categorization of Email into Folders \cite{RON04}}
\begin{my_itemize}
  \item Bag-of-words document representation: messages are represented as 
  vectors of word counts.
  \item Words are downcased.
  \item 100 most frequent words and words that appear only once in the training 
  set are removed, and the remaining words are counted in each message to compose a vector.
  \item In future work, richer representations will be considered, including the following:
    \begin{itemize}
      \item different sections of each email can be treated differently. For example, 
      the system could create distinct features for words appearing in the header, 
      body, signature, attachments, ...etc;
      \item named entities may be highly relevant features.
    \end{itemize}
\end{my_itemize}

\subparagraph{Email Classifications For Contact Centers \cite{ANI03}}
\begin{my_itemize}
  \item Feature sets used for experiments included:
    \begin{itemize}
      \item non-inflected words;
      \item noun phrases;
      \item verb phrases;
      \item punctuation;
      \item length of the Email;
      \item dictionaries.
    \end{itemize}
\end{my_itemize}

\subparagraph{Using GNUsmail to Compare Data Stream Mining Methods for On-line Email \cite{JOSE11}}
\begin{my_itemize}
  \item The main feature of the text preprocessing module is a multi-layer filter 
  structure, responsible for performing feature extraction tasks.
  \item The Inbox and Sent folders are skipped in the learning process because 
  they can be thought of as non-specific folders.
  \item Every mail belonging to any other folder (that is, to any topical folder) 
  goes through a pipeline of linguistic operators which extract relevant features 
  from it.
  \item As the number of possible features is prohibitively large, only the most 
  relevant ones are selected.
\end{my_itemize}

\subparagraph{Content Based Email Classification System by applying Conceptual Maps \cite{BASKARAN09}}
\begin{my_itemize}
  \item Unstructured text: consists of fields like the subject and body.
  \item Categorical text: includes fields such as ``to'' and ``from''.
  \item Numeric data: includes such features as the message size, number of
  recipients and counts of particular characters.
\end{my_itemize}

\newpage
\subsection{Chronological sort of classification papers}
\subparagraph*{2011}
\begin{my_itemize}
  \item Using GNUsmail to Compare Data Stream Mining Methods for On-line Email Classification \cite{JOSE11}.
  \item Enterprise Email Classification Based on Social Network Features \cite{MIN11}.
  \item Automatically tagging email by leveraging other users folders \cite{YEHUDA11}.
\end{my_itemize}

\subparagraph*{2010}
\begin{my_itemize}
  \item An Object Oriented Email Clustering Model Using Weighted Similarities between Emails Attributes \cite{NARESH10}.
  \item A Graph-Based Approach for Multi-Folder Email Classification \cite{sift02}.
\end{my_itemize}

\subparagraph*{2009}
\begin{my_itemize}
  \item Content Based Email Classification System by applying Conceptual Maps \cite{BASKARAN09}.
  \item Email Classification: Solution with Back Propagation Technique \cite{mous05}.
\end{my_itemize}

\subparagraph*{2008}
\begin{my_itemize}
  \item A new approach to Email classification using Concept Vector Space Model \cite{CHAO08}.
  \item E-Classifier: A Bi-Lingual Email Classification System \cite{NOUF08}.
  \item Ontology based classification and categorization of email \cite{BALAKUMAR08}.
  \item Applying Machine learning Algorithms for Email Management \cite{mous03}.
\end{my_itemize}

\subparagraph*{2007}
\begin{my_itemize}
  \item Email Categorization Using Multi-Stage Classification Technique \cite{MD07}.
\end{my_itemize}

\subparagraph*{2005}
\begin{my_itemize}
  \item An Email Classification Model Based on Rough Set Theory \cite{WENQING05}.
  \item eMailSift: Email Classification Based on Structure and Content \cite{sift01}.
\end{my_itemize}

\subparagraph*{2004}
\begin{my_itemize}
  \item Automatic Categorization of Email into Folders \cite{RON04}.
  \item Co-training with a Single Natural Feature Set Applied to Email Classification \cite{mous04}.
\end{my_itemize}

\subparagraph*{2003}
\begin{my_itemize}
  \item Email Classifications For Contact Centers \cite{ANI03}.
\end{my_itemize}


\subsection{Conclusion}
From the research papers related to email classification we conclude the following:
\begin{my_itemize}
    \item SVM Achieved the best results in most papers, but it is computationally 
    expensive and has the hardest implementation;
    \item the basic form of Na\"{\i}ve Bayes algorithm has the simplest implementation 
    but has very low classification accuracy compared to other algorithms;
    \item on-line learning techniques are still under development, they are very 
    hard to implement but characterized by their ability to classify new coming 
    email without rebuilding the model;
    \item off-line learning techniques are used in most papers;
    \item Enron dataset \cite{ENRON} is the most commonly used.
\end{my_itemize}
%=======================================================================================
\section{Comparison Between Related Applications}
\label{sec:related_apps}

\subsection{Outlook rules}
``A rule is an action that Microsoft Outlook performs automatically upon incoming or outgoing 
messages, based on conditions that the user have specified. The user can create a rule 
from a template, from a message, or using his own conditions. Rules fall into one of 
two categories - organization and notification. Rules don't operate on messages that 
have been read, only on those that are unread.'' \cite{OUTLOOK_REF}

\subsection{Gmail Filters}
Gmail's filters allow the user to manage the flow of incoming messages. Using filters, 
the user can label the email, and also can create a set of deterministic rules, and 
based on this; the email is automatically labeled \cite{GMAIL_FILTERS}.

\subsection{Yahoo Filters}
Yahoo Mail helps the user to create personal folders to organize his messages. 
Creating the filter is based on a set of deterministic rules to automatically deliver 
incoming messages to the desired folder \cite{YAHOO_FILTERS}.

\subsection{POPFile}
POPFile is a free, open source, cross-platform mail filter originally written in 
Perl by John Graham-Cumming and maintained by a team of volunteers. It uses a 
naive Bayes classifier to filter mail. This allows the filter to ``learn'' and 
classify mail according to the user's preferences. Typically it is used to filter 
spam mail. It can also be used to sort mail into other user defined ``buckets'' 
or categories - for example, the user may define a bucket into which work email 
is sorted.

The program works in several different modes. In the most popular mode, it sets 
itself up as a proxy between the email client and the POP3 server. As mail is 
downloaded via POP3, the filter identifies and classifies mail and makes a user 
defined modification to the subject line, appending the name of the appropriate 
bucket. The user then sets up rules in the mail client to sort the mail based 
on the subject line modification. An HTML based interface can be used to instruct 
POPFile, allowing users to correct errors in classifications and thus train the 
system to be sensitive to the user's specific requirements.

As an alternative to the subject-line modification (or as a supplement to it), 
the system can also be configured to use custom mail headers instead.

In another possible mode, POPFile can work as an IMAP client that monitors an 
IMAP server for incoming mail and also for messages moved by the user. Incoming 
emails are categorized and then immediately moved to the folder corresponding 
to the categorization. To train POPFile in this mode, the user only needs to 
move the message to the correct folder, i.e. to the folder where POPFile should 
have moved the message.

\subsection{Desirable Features}
\label{desirable_features}
From above, it could be stated that smart email should implement the desirable features enumerated below:
\begin{enumerate}
  \item classifying emails based on deterministic rules;
  \item classifying emails based on dynamic features;
  \item supporting on-line learning classification algorithms;
  \item providing server-side web service for email classification;
  \item supporting several classification algorithms;
  \item supporting multi-label email classification;
  \item providing email summarization service.
\end{enumerate}


\subsection{Summary of Related Applications}
Table \ref{related_applications_summary} summarizes the features of applications related to smart email.

\begin{center}
  \begin{table}[H]
    \begin{tabular}{ | p{3cm} | c | c | c | c |}
      \hline
      Feature/Project              & Outlook Rules \cite{OUTLOOK_REF} & Gmail Filters \cite{GMAIL_FILTERS} & 
                        Yahoo Filters \cite{YAHOO_FILTERS} & PopFile \cite{POPFILE} \\ \hline
      Deterministic Rules  &    Yes        &    Yes        &    Yes      &    Yes  \\ \hline     
      Dynamic Features &    No        &    No         &    No        &    Yes  \\ \hline
      On-line Learning &    No        &    No         &    No        &    Yes  \\ \hline
      Server Side      &    No        &    Yes        &    Yes       &    No   \\ \hline
      Several Classification Algorithms &    No        &    No &    No       &    No   \\ \hline
      Multi-Label support &    No        &    No &    No       &    No   \\ \hline
      Summarization support&    No        &    No &    No       &    No   \\ \hline
    \end{tabular}
    \caption[Summary of related work]{Summary of related applications}
    \label{related_applications_summary}
  \end{table}
\end{center}  


%=======================================================================================
\section{Need To Extend Related Applications}
\label{sec:need_to_extend}
Close study of table \ref{related_applications_summary} reveals the need to extend all existing applications to cover the desirable features enumerated in subsection \ref{desirable_features}

The scope of this project is to design and implement the first five features and the rest will be left for future extensions.

\begin{center}
    \begin{table}[H]
      \begin{tabular}{ | p{3cm} | p{2cm} | p{2cm} | p{2cm} | p{2cm} | p{2cm} |}
        \hline
        Feature/Project              & Outlook Rules \cite{OUTLOOK_REF} & Gmail Filters \cite{GMAIL_FILTERS} & 
        Yahoo Filters \cite{YAHOO_FILTERS} & PopFile \cite{POPFILE} & Smart Email\\ \hline
        Static Features  &    Yes        &    Yes        &    Yes      &    Yes  & \cellcolor[gray]{0.9}Yes \\ \hline     
        Dynamic Features &    No        &    No         &    No        &    Yes  & \cellcolor[gray]{0.9}Yes  \\ \hline
        On-line Learning &    No        &    No         &    No        &    Yes  & \cellcolor[gray]{0.9}Yes \\ \hline
        Server Side      &    No        &    Yes        &    Yes       &    No   & \cellcolor[gray]{0.9}Yes\\ \hline
        Several Algorithms &    No        &    No &    No       &    No   & \cellcolor[gray]{0.9}Yes\\ \hline
        Multi-Label support &    No        &    No &    No       &    No  & \cellcolor[gray]{0.9}No \\ \hline
        Summarization support&    No        &    No &    No       &    No & \cellcolor[gray]{0.9}No  \\ \hline
      \end{tabular}
      \caption[Comparison between Smart Email and related applications]
      {Comparison between Smart Email and related applications}
    \end{table}
\end{center}  

%=======================================================================================
\section{Conclusion}
\label{sec:conclusion_2}
In this chapter, text and email classification were explained and many related techniques 
and technologies were discussed. Also related applications to Smart Email were introduced.

In the next chapter, the main features used for email classifier will be introduced. 
These features are based on the related work presented in this chapter.

