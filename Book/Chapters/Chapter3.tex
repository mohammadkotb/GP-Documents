\label{sec:3_classifier_features}
\label{sec:3_detailed_classifier_features}
\label{sec:3_classifier_tuple}
\label{sec:conclusion_3}
% Chapter 3

\chapter{The Suggested Features Of Smart Email Classifier} % Main chapter title

\label{Chapter3} % For referencing the chapter elsewhere, use \ref{Chapter3} 

\lhead{Chapter 3. \emph{The Suggested Features of Smart Email Classifier}} % This is for the header on each page - perhaps a shortened title

%----------------------------------------------------------------------------------------

\section{Introduction}
In the previous chapter, a background about text and email classification was 
presented. Also, technologies that are related to text classification were 
introduced. Some related applications that are similar to Smart Email with their 
features were discussed.

In this chapter, the suggested features for smart email classifier are discussed.
In section \ref{sec:3_classifier_features}, the features used for email 
classification is presented. Section \ref{sec:3_detailed_classifier_features} 
includes the detailed classifier features. Section \ref{sec:3_classifier_tuple} 
presents the tuple of classifier features used as an input for the email 
classifier. The Chapter is concluded in section \ref{sec:conclusion_3}.
%==============================================================================
\newpage
\section {Classifier Features}
\label{sec:3_classifier_features}
\begin{longtable}{|>{\centering}p{2.5cm}|>{\centering}p{3cm}|>{\centering}p{3cm}|>{\centering}p{3cm}|}
\hline
\multirow{19}{2.5cm}{Features needed for the email classifier (Automatic Categorization
of emails into folders)}
 & \multicolumn{3}{c|}{Features}\tabularnewline
\cline{2-4}
\cline{2-4} 
 & Email & Receiver & Attachment\tabularnewline
\cline{2-4} 
 & Email ID \cite{Anatomy00} & Domain of receiver \cite{Carmona2011} \cite{MANUEL11} &  Attachment  type \cite{Carmona2011} \cite{MANUEL11}\tabularnewline
\cline{2-4} 
 & Body Length \cite{Carmona2011} \cite{MANUEL11} & Number of CC \cite{Carmona2011} \cite{MANUEL11} & Has an attachment \cite{Carmona2011} \cite{MANUEL11}\tabularnewline
\cline{2-4} 
 & Content Type \cite{Anatomy00} & Number of receivers \cite{Carmona2011} \cite{MANUEL11} & Number of attachments \cite{Carmona2011} \cite{MANUEL11}\tabularnewline
\cline{2-4} 
 & Domain of sender \cite{Carmona2011} \cite{MANUEL11} & Number of To \cite{Carmona2011} \cite{MANUEL11} & \tabularnewline
\cline{2-4} 
 & Email Date \cite{KIRI2004} \cite{Anatomy00} & Receiver Username \cite{Carmona2011} \cite{MANUEL11} & \tabularnewline
\cline{2-4} 
 & Email Sender \cite{Carmona2011} \cite{RON04} \cite{Anatomy00} \cite{MANUEL11} &  & \tabularnewline
\cline{2-4} 
 & Email Signature \cite{MANUEL11} &  & \tabularnewline
\cline{2-4} 
 & Email Subject \cite{Carmona2011} \cite{RON04} \cite{MANUEL11} &  & \tabularnewline
\cline{2-4} 
 & Is Bcc \cite{Carmona2011} \cite{RON04} \cite{MANUEL11} &  & \tabularnewline
\cline{2-4} 
 & Is distribution List \cite{Carmona2011} \cite{MANUEL11} &  & \tabularnewline
\cline{2-4} 
 & Language \cite{Carmona2011} \cite{MANUEL11} &  & \tabularnewline
\cline{2-4} 
 & MIME Version \cite{Anatomy00} &  & \tabularnewline
\cline{2-4} 
 & Number of punctuation Letters \cite{Carmona2011} \cite{MANUEL11} &  & \tabularnewline
\cline{2-4} 
 & Percentage of capital letters \cite{Carmona2011} \cite{MANUEL11} &  & \tabularnewline
\cline{2-4} 
 & Sender Username \cite{Carmona2011} \cite{MANUEL11} &  & \tabularnewline
\cline{2-4} 
 & Subject Length \cite{Carmona2011} \cite{MANUEL11} &  & \tabularnewline
\cline{2-4} 
 & Wordgram Frequency \cite{Carmona2011} \cite{RON04} \cite{MANUEL11} &  & \tabularnewline
\hline
\end{longtable}
%----------------------------------------------------------------------
\section {Detailed Classifier Features}
\label{sec:3_detailed_classifier_features}

\begin{longtable}{|>{\centering}p{2cm}|>{\centering}p{2.5cm}|>{\centering}p{3cm}|>{\centering}p{3cm}|>{\centering}p{3cm}|}
\hline 
Category & Feature & Description & Values & Source (preparation)\tabularnewline
\hline
\hline
Email & Email ID \cite{Anatomy00} & Identifier for the email message & Long & System maintained primary key\tabularnewline
\cline{2-5}
 & Domain of sender \cite{Carmona2011} \cite{MANUEL11} & Mail Service Provider (gmail.com, hotmail.com, ..etc) & String & Obtained from the sender's email, by taking the substring after the
'@' character\tabularnewline
\cline{2-5}
 & Language \cite{Carmona2011} \cite{MANUEL11} & Dominant language in the email body & String & Use a special module to detect the language type of the email body\tabularnewline
\cline{2-5}
 & Email Sender \cite{Carmona2011} \cite{RON04} \cite{Anatomy00} \cite{MANUEL11} & Email address of the sender & String & Obtained directly from the email header\tabularnewline
\cline{2-5}
 & Content Type \cite{Anatomy00} & Content type  & String & Obtained directly from the email header\tabularnewline
\cline{2-5}
 & Email Date \cite{KIRI2004} \cite{Anatomy00} & Date of sending the email represented as the number of milliseconds
since January 1, 1970, 00:00:00 GMT & Long & The date is obtained directly from the email header and then transformed
to the long representation\tabularnewline
\cline{2-5}
 & MIME Version \cite{Anatomy00} & MIME is an internet standard to extend the format of the email to
support non-ASCII data & Integer & Obtained directly from the email header\tabularnewline
\cline{2-5}
 & Bcc \cite{Carmona2011} \cite{RON04} \cite{MANUEL11} & List of email receivers as Bcc & Each recipient is represented as a boolean attribute in the feature
tuple. & Obtained directly from the email header \tabularnewline
\cline{2-5}
 & Number of punctuation Letters \cite{Carmona2011} \cite{MANUEL11} & Number of punctuation characters in the body & Integer & Count the number of punctuation letters in the email body\tabularnewline
\cline{2-5}
 & Is distribution List \cite{Carmona2011} \cite{MANUEL11} & Flag to indicate whether the client received this email from a group/distribution
list or not & Boolean & Obtained directly from email header\tabularnewline
\cline{2-5}
 & Email Signature \cite{MANUEL11} & Signature of the email sender, at the end of the email & String & The signature is extracted from email body\tabularnewline
\cline{2-5}
 & Wordgram Frequency \cite{Carmona2011} \cite{RON04} \cite{MANUEL11} & Email Wordgram Frequency & Integer & Count the number of wordgrams in the email\tabularnewline
\cline{2-5}
 & Subject Length \cite{Carmona2011} \cite{MANUEL11} & Length of the email subject & Integer & Calculate the size of the subject string\tabularnewline
\cline{2-5}
 & Percentage of capital letters \cite{Carmona2011} \cite{MANUEL11} & Percentage of the capital letters to the letters in the email body & Double & Count the number of capital letters and divide it by the sum of the
sizes of all ASCII words in the email body\tabularnewline
\cline{2-5}
 & Body Length \cite{Carmona2011} \cite{MANUEL11} & Size of the email body & Integer & Calculate the size of the body string\tabularnewline
\cline{2-5}
 & Sender Username \cite{Carmona2011} \cite{MANUEL11} & Name of Sender & String & Obtained directly from email header\tabularnewline
\cline{2-5}
 & Total Number of words & Number of words in the email body & Integer & Calculate the number of words in the email body\tabularnewline
\cline{2-5}
 & Email Subject \cite{Carmona2011} \cite{RON04} \cite{MANUEL11} & Subject of the email & String & Obtained directly from the email header\tabularnewline
\hline 
Receiver & Receiver Username \cite{Carmona2011} \cite{MANUEL11} & Name of receiver & String & Obtained directly from email header\tabularnewline
\cline{2-5}
 & Number of receivers \cite{Carmona2011} \cite{MANUEL11} & Number of email receivers & Integer & Count the number of receivers obtained from email header\tabularnewline
\cline{2-5}
 & Number of CC \cite{Carmona2011} \cite{MANUEL11} & Number of CC recipients & Integer & Count the number of CC recipients obtained from email header\tabularnewline
\cline{2-5}
 & Number of To \cite{Carmona2011} \cite{MANUEL11} & Number of CC & Integer & Count the number of receivers mentioned in the TO header\tabularnewline
\cline{2-5}
 & Domain of receiver \cite{Carmona2011} \cite{MANUEL11} & Mail Service Provider(s) for the receiver(s) & String & Obtained directly from email header\tabularnewline
\hline 
Attachment & Number of attachments \cite{Carmona2011} \cite{MANUEL11} & Number of attached files in the email  & Integer & Count the number of attachments obtained from the IMAP interface\tabularnewline
\cline{2-5}
 &  Attachment  type \cite{Carmona2011} \cite{MANUEL11} & Type of attachment & String & 	Has an attachment \cite{Carmona2011} \cite{MANUEL11}	Flag to denote whether the email
has an attachment	Boolean 	If the number of attachment is zero, return
false, else return true
\tabularnewline
\hline
\end{longtable}
%----------------------------------------------------------------------

%-------------------------------------------------------------------------
\newpage
\section {Tuple of the classifier Features (Classifier Input)}
\label{sec:3_classifier_tuple}

\begin{center}
\begin{tabular}{|c|}
\hline 
email\_id\tabularnewline
\hline
date\tabularnewline
\hline 
sender\_email\tabularnewline
\hline 
sender\_username\tabularnewline
\hline 
Domain of sender\tabularnewline
\hline 
is\_bbc\tabularnewline
\hline 
subject\tabularnewline
\hline 
Subject\_length\tabularnewline
\hline 
content\tabularnewline
\hline 
content\_mime\_version\tabularnewline
\hline 
body\_length\tabularnewline
\hline 
signature\tabularnewline
\hline 
number\_of\_receivers\tabularnewline
\hline 
Percentage\_of\_capital\_letters\tabularnewline
\hline 
Number\_of\_punctuation\_letters\tabularnewline
\hline 
language\tabularnewline
\hline 
has\_attachments\tabularnewline
\hline 
number\_of\_attachments\tabularnewline
\hline
\end{tabular}
\end{center}

\newpage


\section{Conclusion}
\label{sec:conclusion_3}
In this chapter the main features used for email classification was introduced.
At the end of the chapter, the tuple of the classifier features was presented.
Also it has been decided to focus on the following features:
\begin{my_itemize}
  \item static features;
  \item dynamic features
  \item on-line learning;
  \item server side;
  \item several algorithms.
\end{my_itemize}
And left the remaining features as a feature work, as shown in the following 
table.

In next chapter, project conceptualization and the proposed workflow will be 
discussed.

\begin{center}
    \begin{table}[H]
      \begin{tabular}{ | p{3cm} | p{2cm} | p{2cm} | p{2cm} | p{2cm} | p{2cm} |}
        \hline
        F/P              & Outlook Rules \cite{OUTLOOK_REF} & Gmail Filters \cite{GMAIL_FILTERS} & 
        Yahoo Filters \cite{YAHOO_FILTERS} & PopFile \cite{POPFILE} & Smart Email\\ \hline
        Static Features  &    Yes        &    Yes        &    Yes      &    Yes  & \cellcolor[gray]{0.9}Yes \\ \hline     
        Dynamic Features &    No        &    No         &    No        &    Yes  & \cellcolor[gray]{0.9}Yes  \\ \hline
        On-line Learning &    No        &    No         &    No        &    Yes  & \cellcolor[gray]{0.9}Yes \\ \hline
        Server Side      &    No        &    Yes        &    Yes       &    No   & \cellcolor[gray]{0.9}Yes\\ \hline
        Several Algorithms &    No        &    No &    No       &    No   & \cellcolor[gray]{0.9}Yes\\ \hline
        Multi-Label support &    No        &    No &    No       &    No  & \cellcolor[gray]{0.9}No \\ \hline
        Summarization support&    No        &    No &    No       &    No & \cellcolor[gray]{0.9}No  \\ \hline
      \end{tabular}
      \caption[Comparison between Smart Email and related applications]
      {Comparison between Smart Email and related applications}
    \end{table}
\end{center}  
