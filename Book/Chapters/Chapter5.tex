% Chapter 5

\chapter{Development Process, Environment and Tools} % Main chapter title

\label{Chapter5} % For referencing the chapter elsewhere, use \ref{Chapter5} 

\lhead{Chapter 5. \emph{Development Process, Environment and Tools}} % This is for the header on each page - perhaps a shortened title

\section{Introduction}
In Chapter \ref{Chapter4}, Smart Email requirements and design were discussed. In this chapter, 
the adopted process is examined in detail; the stages of work, the tools used 
and the work environment.

The environment needed to be set up is discussed in section \ref{sec:5_setting_up_env}.
In section \ref{sec:5_sw_eng_process_followed}, 
the software engineering model followed throughout the project will be explained. 
In section \ref{sec:5_dev_process}, the development process will be described in detail. In section \ref{sec:5_tools_used}, 
all the tools used will be explained. The chapter is concluded in section \ref{sec:conclusion_5}.
%----------------------------------------------------------------------------------------
\section{Setting Up The Environment}
\label{sec:5_setting_up_env}
The needed environment for the project to run
\begin{itemize}
  \item Any operating system that is compatible with Java Runtime Environment.
  \item Java Runtime Environment.
  \item Tomcat 6 server.
  \item MySQL and PostgreSQL database.
  \item RubyOnRails Framework \cite{ROR}.
\end{itemize}

%--------------------------------------------------------
\section{Software Engineering Process Followed}
\label{sec:5_sw_eng_process_followed}
Since the project included many different processes and stages. A long time is 
spent and multiple team members were working on it. It was critical to follow 
a standard methodological software engineering model to keep the work organized 
and fully utilize the effort.

Software Process Improvement (SPI) for Small Medium Enterprises (SMEs)
model is a model developed by software Engineering Competence Center
whose goal is to help small and medium enterprise to raise the quality of
their products by using modern software development processes and practices. 
It consists of several processes: management, development, peer-review, 
quality assurance and configuration management processes. Since
the SPI model targets mainly enterprise and business, it was required to
slightly modify it to fit the scientific nature of the project. Three of these 
processes were followed in the work which are the development, peer review and
configuration management processes.

\begin{itemize}
  \item Development Process: the development processes contained several
  phases so it will be discusses in the next section \ref{sec:5_dev_process} .
  \item Peer review Process: peer review allows to detect the errors early and
  almost all team members share their thoughts in the same point and
  look at it from many points of view. Throughout the project, all work
  was reviewed among the team members.
  \item Configuration Management Process: concerned with controlling and
  reporting changes during the project's lifetime. During the implementation 
  phase, the source code was edited frequently and by multiple
  team members. So configuration management was applied by using
  source control system.
\end{itemize}

%--------------------------------------------------------
\section{Development Process}
\label{sec:5_dev_process}

The development process adopted throughout the project included the following
phases: requirements specification, planning, analysis, implementation, integration,
 testing and deployment.

\subsection{Requirements Specification}
Extracting the requirements is considered one of the most important phases
in any software development. The main requirement was a classification web 
service with a REST API \cite{REST}.
\subsection{Planning}
To achieve the main requirement, several sub-requirements were planned.
First, it was important to compare between classification algorithms and choose 
a subset of them to use, then choosing the best Email feature combination to 
give the highest accuracy, then choosing the API syntax for the classification 
web service. Finally, building a web monitoring web service and a browser extension 
to demonstrate the capabilities of the classification web service.
\subsection{Analysis}
As stated in the planning phase, classification algorithms were analyzed and 
several email features were compared. The results of these comparisons are 
discussed in chapter 6.

\subsection{Design}
Since several algorithms and features combinations are analyzed and compared, 
a flexible design that enables changing the classification algorithm or the 
classification features was needed. Design is described in detail in the previous 
chapter.

\subsection{Implementation}
The implementation phase was the most time demanding phase in the project. The 
implementation phase was divided into 3 parts. First implementing the 
classification web service using JavaEE and REST API \cite{REST},
then implementing the email monitoring web service using ruby on rails \cite{ROR}
and finally the browser extension using Google Chrome browser extension API \cite{CHROME}.

\subsection{Integration}
After implementing the classification web service,
it was required to integrate the email monitoring web service and the browser extension with
the classification web service and make sure they can communicate with the classification web
service using the REST API.

\subsection{Testing}
Several levels and types of testing were applied throughout the project.
\begin{itemize}
  \item Unit testing was applied on several standalone modules.
  \item Integration testing was applied to the classification service as a whole.
\end{itemize}

\subsection{Deployment}
The classification web service was configured so that it can be deployed on any 
web server that supports javaEE. The email monitoring web service was deployed 
to Heroku \cite{HEROKU}. The Chrome browser extension can be deployed to Chrome 
extension web store.

%--------------------------------------------------------
\section{Tools Used}
\label{sec:5_tools_used}

\subsection{Programming Languages Used}
\begin{itemize}
  \item Java Enterprise Edition version 6 was used as the main language in the 
  classification web service.
  \item Ruby on Rails \cite{ROR} was used for developing the email monitoring 
  web service.
  \item Javascript was used for developing the browser extension.
\end{itemize}

\subsection{Integrated Development Environments (IDEs)}
Eclipse IDE was used for the classification web server.
Vim text editor was used for developing the browser extension and the email 
monitoring web service.

\subsection{External Libraries}
The following libraries were used
\begin{itemize}
  \item WEKA \cite{WEKA} data-mining library used to provide implementation for different classification algorithms.
  \item Jersy \cite{JERSY} is used for building RESTful web applications in java.
  \item Gmailr \cite{GMAILR} is used to detect events within the Gmail web interface.
\end{itemize}


\subsection{Web Servers}
The following web servers were used
\begin{itemize}
  \item Tomcat 6 is used as a java web server for the classification web service.
  \item Rails web server is used for the email monitoring web service.
\end{itemize}

\subsection{Other Tools}
The following libraries were used
\begin{itemize}
  \item Dia Diagram : is used to draw UML and other diagrams.
  \item Matplotlib python graph module is used for drawing graphs and charts in Chapter 6.
\end{itemize}

\subsection{Source Control}
git source control system \cite{GIT} was chosen to allow versioning and provide 
the ability to rollback to any version of the project.
%--------------------------------------------------------
\section{Conclusion}
\label{sec:conclusion_5}
In this chapter, the development process, used tools and project environment were 
discussed in depth. In the next chapter, implementation details and analysis for the classification 
procedure will be provided.
