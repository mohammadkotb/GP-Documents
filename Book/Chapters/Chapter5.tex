% Chapter 5

\chapter{Development Process, Environment and Tools} % Main chapter title

\label{Chapter5} % For referencing the chapter elsewhere, use \ref{Chapter5} 

\lhead{Chapter 5. \emph{Development Process, Environment and Tools}} % This is for the header on each page - perhaps a shortened title

\section{Introduction}
In Chapter 4, Smart Email architecture and design was discussed. In This Chapter, 
the adopted process is examined in details; the stages of work, the tools used 
and the work environment.

The environments needed to be set up are discussed in section 5.2. In section 5.3, 
the software engineering model followed throughout the project will be explained. 
In section 5.4, the development process will be described in details. In section 5.5, 
all the tools used will be explained. In section 5.6 the hardware and software 
environments will be described in details.
%----------------------------------------------------------------------------------------
\section{Setting Up The Environment}
The needed Environments for the project to run
\begin{itemize}
\item Any operating system that is compatible with java Runtime Environment.
\item Java runtime environment.
\item Tomcat 6 server.
\item MySql database.
\end{itemize}

%--------------------------------------------------------
\section{Software Engineering Process Followed}
Since the project included many different processes and stages. A long time is spent and multiple team members were working on it. It was critical to follow a standard methodological software engineering model to keep the work organized and fully utilize the effort.


Software Process Improvement (SPI) for Small Medium Enterprises (SMEs)
model is a model developed by software Engineering Competence Center
whose goal is to help small and medium enterprise to raise the quality of
their products by using modern software development processes and practices. It consists of several processes: the management, development, peer-review, quality assurance and configuration management processes. Since
the SPI model targets mainly enterprise and business, it was required to
slightly modify it to fit scientific natural of the project. Three of these processes were followed in the work which are the development, peer review and
configuration management processes.

\begin{itemize}
\item Development Process: the development processes contained several
phases so it will be discusses in next section 5.4 .
\item Peer review Process: peer review allows to detect the errors early and
almost all team members share their thoughts in the same point and
look at it from many points of view. Throughout the project, all work
was reviewed among the team members.
\item Configuration Management Process: concerned with controlling and
reporting changes during the project's lifetime. During the implementation phase, the source code was edited frequently and by multiple
team members. So configuration management was applied by using
source control system.
\end{itemize}

%--------------------------------------------------------
\section{Development Process}

The development process adopted through the project included the following
phases: requirements specification, planning.

\subsection{Requirements Specification}
Extracting the requirements is considered one of the most important phases
in any software development. The main requirement was a classification web service with a REST api \cite{REST}.
\subsection{Planning}
To achieve the main requirement, several sub-requirements were planned.
First, it was important to compare between classification algorithms and choose a subset of them to use,
then choosing the best Email feature combination to give the highest accuracy,
then choosing the api syntax for the classification web service.
Finally, building a web monitoring web service and a browser extension to demonstrate the capabilities of the classification web service.
\subsection{Analysis}
As stated in the planning phase, Classification algorithms were analyzed and several Email features were compared ,
the results of this comparisons are discussed in chapter 6.
Enron dataset was used to do the analysis. The largest 7 users were chosen to do the analysis \cite{RON04}.

\subsection{Design}
Since several algorithms and features combinations are analyzed and compared , a flexible design that enables changing
the classification algorithm or the classification feature were needed.
Design is described in details in the previous chapter.

\subsection{Implementation}
The implementation phase was the most time demanding phase in the
project.
The implementation phase was divided into 3 parts.
First implementing the classification web service using JavaEE and REST api \cite{REST},
then implementing the email monitoring web service using ruby on rails \cite{ROR}
and finally the browser extension using google chrome browser extension api \cite{CHROME}.
\subsection{Integration}
After implementing the classification web service,
it was required to integrate the email monitoring web service and the browser extension with
the classification web service and make sure they can communicate with the classification web
service using the REST api.

\subsection{Testing}
Several levels and types of testing were applied throughout the project.
\begin{itemize}
\item Unit testing was applied on several standalone modules.
\item Integration testing was applied to the classification service as a whole.
\end{itemize}
\subsection{Deployment}
The classification web service was configured so that it can be deployed on any web server that supports javaEE.
The email monitoring web service was deployed to heroku \cite{HEROKU}.
The chrome browser extension can be deployed to chrome extension web store.

%--------------------------------------------------------
\section{Tools Used}

\subsection{Programming Languages Used}
\begin{itemize}
\item Java Enterprise Edition version 6 was used as the main language in the classification web service.
\item Ruby on rails \cite{ROR} was used for developing the email monitoring web service.
\item Javascript was used for developing the browser extension.
\end{itemize}

\subsection{Integrated Development Environments (IDEs)}
Eclipse IDE was used for the classification web server.
Vim text editor was used for developing the browser extension and the email monitoring web service.
\subsection{Web Servers}
The following web servers were used
\begin{itemize}
\item Tomcat 6 is used as a java web server for the classification web server.
\item Rails web server as a server for the email monitoring web server.
\end{itemize}

\subsection{Source Control}
git source control system \cite{GIT} was chosen to allow versioning and provide the ability to rollback to any version of the project.
%--------------------------------------------------------
\section{Hardware And Software Environment}

%--------------------------------------------------------
\section{Conclusion}
In this chapter, the development process, used tools and project environment were discussed in depth. In the next chapter, an analysis for the classification procedure is provided.
