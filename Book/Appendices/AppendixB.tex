% Appendix Template

\chapter{Technical Background for Email Monitoring Web Service} % Main appendix title

\label{AppendixC} % Change X to a consecutive letter; for referencing this appendix elsewhere, use \ref{AppendixX}

\lhead{Appendix C. \emph{Technical Background for Email Monitoring Web Service}} % Change X to a consecutive letter; this is for the header on each page - perhaps a shortened title



\section{Ruby on Rails Web Framework}
\begin{itemize}
 \item Ruby On Rails is an open source full stack web application framework the ruby 
      programming language. Like many other frameworks, Rails uses the 
      Model/View/Controller (MVC) architecture pattern to organize application 
      programming which makes it easier to develop, maintain and deploy web applications.
  \item Smart Email application uses Rails 3.2 and Ruby 1.9.2
\end{itemize}

\section{Heroku Deployment Server}
Heroku is a cloud application platform which provides a platform as a service for 
building, deploying and running cloud applications. Deployment is done by pushing 
to a git repository for continuous deployment.

Also an addon named logentries integrated with heroku is used, this addon is a 
complete log management system for  giving alerts when serious errors occured to 
the application, logging all applications events and requests and this beside 
the normal log management features such as Tail, Search, Visualization, and Notifications.


\section{Database}
\begin{itemize}
 \item Production:
  \begin{itemize}
   \item PostgreSQL: it is a powerful open source object relational database system 
      with architecture that have proven reliability, data integrity, and correctness. 
      PostgreSQL runs on almost all major platforms including Linux, Windows and Mac OS.
  \end{itemize}
 \item Development:
 \begin{itemize}
  \item MySQL: it is the world’s most popular open source database because of it’s performance,
      high reliability and ease of use. MySQL runs on more than 20 platforms including Linux, Windows and Mac OS.
 \end{itemize}
\end{itemize}

\section{RESTful Web Service}
REST is a style of software architecture for distributed systems which uses noun and verbs to describe 
web pages. Rest describes six constraints applied to the architecture:
\begin{itemize}
 \item Client-Server: a uniform interface which separate server from client and accessed with HTTP Get, Post, Put, and Delete.
 \item Stateless.
 \item Cache-able.
 \item Layered System.
 \item Named Resources.
 \item Hyperlinks.
\end{itemize}

Also REST have some advantages over SOAP:
\begin{itemize}
 \item Simpler.
 \item Caching.
 \item Easier state transitions with hyperlinks and no intermediate stubs as in SOAP i.e data goes directly from client to server.
\end{itemize}

\section{Resque-Redis Job Scheduler}
\begin{itemize}
 \item Redis is an open source, advanced key-value store which is often refers to as a data structure server.
 \item Resque is a redis backed library for creating background jobs, placing these jobs in queues and 
      later processing them. Resque queues are persistent, provide visibility to their content and 
      store jobs as simple JSON packages.
 \item Resque is used  in Smart Email to schedule the monitor job which monitors the users emails every 30 minutes. 
\end{itemize}

\section{IMAP}
\begin{itemize}
 \item An application layer internet protocol that allows an email client to access emails on a remote mail server.
 \item IMAP is used to access users email accounts in the classification web service and application web service.
 \item IMAP-IDLE:
 \begin{itemize}
  \item IMAP feature which enables the email client connected to the mail server to be notified when a new email arrives.
  \item It is used in Smart Email to monitor users emails to be able to send the email to the classification service as soon as the email arrives.
 \end{itemize}
\end{itemize}
