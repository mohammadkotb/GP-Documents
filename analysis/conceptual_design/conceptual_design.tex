\documentclass[a4paper,10pt]{article}
\usepackage[utf8x]{inputenc}
\usepackage{multirow}
\usepackage[pdftex]{graphicx}
\usepackage{float}
\newenvironment{my_itemize}
{\begin{itemize}
  \setlength{\itemsep}{0cm}
  \setlength{\parskip}{0cm}}
{\end{itemize}}
\newenvironment{my_enumerate}
{\begin{enumerate}
  \setlength{\itemsep}{0cm}
  \setlength{\parskip}{0cm}}
{\end{enumerate}}

\newenvironment{my_description}
{\begin{description}
  \setlength{\itemsep}{0cm}
  \setlength{\parskip}{0cm}}
{\end{description}}


\begin{document}

\begin{titlepage}
\vspace{-1.5cm}
\begin{center}
\includegraphics[width=2cm]{Logo_Alexandria_University.jpg}\\
\vspace{1cm}
\textbf{\large ALEXANDRIA UNIVERSITY} \\
\textbf{FACULTY OF ENGINEERING} \\
{\small  COMPUTER AND SYSTEMS ENGINEERING DEPARTMENT}

\vspace{2.5cm}
\textbf{\LARGE Conceptualization Document}\\
\textbf{\small Smart Email Automatic Email Classification and Summarization}\\

\vspace{1cm}
{ Ahmed El-Sharkasy, Ahmed Kotb, Amr Nabil, Mohammad Kotb, Moustafa Mahmoud }
\end{center}

\vspace{1ex}
\textbf{Supervisors:} Prof. Dr. Mohamed Abou-gabal, Dr. Mustafa Elnainany
\end{titlepage}

\newpage
\tableofcontents
\newpage

\section{Introduction}

\subsection{Overview Application}
Email has been an efficient and popular communication mechanism as the number 
of Internet users increases. Therefore, email management has become an important 
and growing problem for individuals and organizations because it is prone to misuse.

One of the problems that are most important is disordered email messages, 
congested and unstructured emails in mail boxes. It may be very hard to 
find archived email messages, search for previous emails with specified contents 
or features when the mails are not well structured and organized. Email 
summarization is another important and challenging problem.

Smart Email is an application addressing these problems by providing automatic 
email summarization and categorization into folders.

%==============================================================================
\subsection{Project Objectives}
The business objectives of this project are listed below. Delivering these 
objectives will deliver the expected benefits of the application.
\begin{my_itemize}
  \item Automatically organizing congested and uncategorized emails into folders.
  \item Summarizing lengthy email threads into a brief summary describing the 
	main points in these threads.
\end{my_itemize}


%==============================================================================
\subsection{Assumptions}
Some assumptions generally must be made in order to write a succinct definition 
of the application. Some assumptions are technical (e.g. ``the new system will 
employ a normalized database schema''), while others are business natured (e.g. 
``the client will provide an enterprise Oracle environment'').  Assumptions 
critical to the success of this project are listed below:
\begin{my_itemize}
  \item each user will have his own trained classification and summarization 
	models based on the user supplied training data;
  \item a web service will be implemented to provide email 
	classification and summarization support for different mail 
	servers (Example: Google, Hotmail, Yahoo);
  \item english language will be supported by default;
  \item email summarization will be provided as on-demand service by the application user;
  \item the servers providing the web service have to provide sufficient 
	processing power for achieving reasonable response time;
  \item email classification will be based on the email subject and content. 
	Other features can be used as well, such as the email sender and receiver. 
	However, email attachment will not be considered as a classification feature for the email.
\end{my_itemize}

%==============================================================================
\subsection{Limitations}
Every project has limitations on the scope of the problem it attempts to solve, 
on the capabilities of the implementation technology, ...etc. Some limitations 
are assumptions on the extent of feature scope. Others are restrictions on resources 
or methods for achieving the objectives. The limitations of this project are listed below:

\begin{my_itemize}
  \item sufficient number of emails is needed for each email category to achieve a reasonable classification accuracy;
  \item emails with size less than a specified threshold will not be summarized.
\end{my_itemize}

%==============================================================================
\subsection{Open items and Risks}
During the conceptualization of the project and in the process of writing this document, 
some issues may have been discovered, and others may remain open.
\begin{my_itemize}
  \item Multi-label classification.
  \item Arabic support.
  \item The need of a client side application to assist the server side.
  \item Online training for classification models.
  \item Specifications for the web server providing the web service.
\end{my_itemize}


%==============================================================================
\section{Proposed Workflow}
\subsection{Overview}
This section includes a concise description of the features and operation of the 
proposed solution. It also compares and contrasts the solution with the current 
process. The context and workflow of the proposed solution are defined in subsequent sections.

Features:
\begin{my_itemize}
  \item automatically categorizing incoming emails into predefined categories;
  \item summarizing long email threads into a brief summary.
\end{my_itemize}


%==============================================================================
\subsection{Context Diagram}
The Context Diagram illustrates the modules, business processes, etc that feed 
or interact with this proposed application.\\ \\
%add photo
\includegraphics[width=13cm]{context_diagram.jpeg}
%==============================================================================
\subsection{High Level Workflow}
The workflow required to complete the primary objectives of the proposed 
application is described below. The workflow is business-centered, and 
includes ``decision forks'' for decisions the business user, or application, 
must make to achieve the objective. The workflow omits application faults or exceptions. 
Individual tasks in the workflow are described.

\begin{my_enumerate}
  \item Workflow/Process Map \\ \\
  % add the photo
  \includegraphics[width=13cm]{workflow_process_map.png}
  \item Workflow Description
  \begin{my_itemize}
    \item Incoming Emails are preprocessed.
    \item Summarization module checks that email length is greater than predefined threshold:
    \begin{my_itemize}
      \item if the length is greater than the threshold, the email is fed to the 
	    summarizer which outputs the summary;
      \item short emails are returned.
    \end{my_itemize}
    \item Classification module uses the user module to classify the email into a specific label.
    \item The result is returned by the web service.
  \end{my_itemize}
\end{my_enumerate}


%==============================================================================
\section{Business Requirements}

This section identifies, enumerates and explores the business requirements that must 
be met by the application. Business requirements include capturing the types of users, 
the basic inputs and outputs, the system's dependencies, and the tasks the system should 
accomplish. Some business requirement for a web-based retail site might include:

\begin{my_description}
  \item[Users] ``The system supports public browsers, registered members, 
	      customer service administrators, and system administrators.''
  \item[Tasks] ``Users have personalized accounts,” “shopping carts expire in a configurable period, and the default is 10 days''.
\end{my_description}

It is important to confirm that Task Requirements include all the tasks required to meet the business objectives.

%==============================================================================
\subsection{Users}
All applications have users and most have several users of different types. This 
section identifies, at a high level, the types of users of the system.

\begin{my_enumerate}
  \item Application User: Requests Classification and summarization of emails.
  \item Admin: Tune classification and summarization algorithms.
\end{my_enumerate}

%==============================================================================
\subsection{Inputs/Outputs}
This section identifies and describes the inputs to and outputs from the new 
application. Inputs can include electronic inputs, like RSS and EDI feeds, 
updates from external databases, etc, as well as human inputs, like 
``user X keys in results from report Y.'' Output can include electronic feeds, 
printed reports, ...etc. In this section, all electronic inputs and outputs 
are captured. Human inputs are omitted from this section.

\begin{my_enumerate}
  \item Inputs
  \begin{my_itemize}
    \item Unclassified Emails.
    \item Classified Emails (training data).
    \item Email Threads.
  \end{my_itemize}
  \item Outputs
  \begin{my_itemize}
    \item Classified emails.
    \item Summarized email threads.
    \item Classification model.
  \end{my_itemize}
\end{my_enumerate}


%==============================================================================
\subsection{Dependencies}
Human inputs are also categorized as dependencies.  List all dependencies.

\begin{my_enumerate}
  \item Request for summarizing an email thread.
  \item Registeration data.
\end{my_enumerate}


%==============================================================================
\subsection{Task Requirements}
Task Requirements correlate to the use cases, above. There should be at least 
one task for each use case bubble. Each use case should be briefly described. 
If there are specific requirements related to the use case, they should also 
be listed. This section captures high-level, basic functional requirements. 
Detailed requirements will be defined during the Requirements Specification contests.

TODO...
%==============================================================================
\subsection{Security Requirements}
This section documents, at a high level, the basic security requirements of the 
application. For example, does the system require the user to log in?  Should the 
user’s identity be authenticated on just this system, or against a central authority?  
Are there any special or unusual security requirements, like fingerprint scanning?

\begin{my_enumerate}
  \item The system requires every user to log in with a unique identity (email).
  \item User Data (email) should be transmitted with a secure connection.
  \item Administrators shouldn’t have access to user data (email).
\end{my_enumerate}



%==============================================================================
\subsection{Performance Requirements}
Performance Requirements for the application are defined at a high level below. 
If there are specific requirements for specific features to perform at a quantifiable 
level, they too are listed below. These requirements will be developed in more detail when 
the Requirements Specification is written, in a later phase.

\begin{my_enumerate}
  \item Web service should respond within a reasonable time.
  \item Initial training phase should finish within a reasonable time.
  \item The system will achieve high uptime.
\end{my_enumerate}


%==============================================================================
\subsection{Data Migration}
Data Migration describes the data that needs to be moved from an older or external 
system to the new system, in order for it to operate at launch. Migration also includes 
data that must be transferred from the new system to another external system. 
Any data migration requirements are listed below.

\begin{my_enumerate}
  \item Users classified emails for training phase \\
	pre-classified emails are needed to build specific user mode, users' emails can be discarded after building the model
\end{my_enumerate}


\end{document}
