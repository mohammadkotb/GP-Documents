\documentclass[12pt]{article}
\usepackage{multirow}
\usepackage[pdftex]{graphicx}
\usepackage{float}


\newenvironment{my_itemize}
{\begin{itemize}
  \setlength{\itemsep}{0cm}
  \setlength{\parskip}{0cm}}
{\end{itemize}}
\newenvironment{my_enumerate}
{\begin{enumerate}
  \setlength{\itemsep}{0cm}
  \setlength{\parskip}{0cm}}
{\end{enumerate}}
\newenvironment{my_desc}
{\begin{description}
  \setlength{\itemsep}{0cm}
  \setlength{\parskip}{0cm}}
{\end{description}}


\begin{document}

\begin{titlepage}
\vspace{-1.5cm}
\begin{center}
\includegraphics[width=2cm]{Logo_Alexandria_University.jpg}\\
\vspace{1cm}
\textbf{\large ALEXANDRIA UNIVERSITY} \\
\textbf{FACULTY OF ENGINEERING} \\
{\small  COMPUTER AND SYSTEMS ENGINEERING DEPARTMENT}

\vspace{2.5cm}
\textbf{\LARGE Software Design Description Document}\\
\textbf{\small Smart Email Automatic Email Classification and Summarization}\\
\vspace{1cm}
{ Ahmed El-Sharkasy, Ahmed Kotb, Amr Nabil, Mohammad Kotb, Moustafa Mahmoud }
\end{center}

\vspace{1ex}
\textbf{Supervisors:} Prof. Dr. Mohamed Abou-gabal, Dr. Mustafa ElNainay
\end{titlepage}

\newpage
\tableofcontents
\newpage

\section{Introduction}

This software design description document is a model of the smart email software system to be created. The model provides the precise design information needed for planning, analysis, and implementation of the software system. The document represents a partitioning of the system into design entities and describe the important properties and relationships among those entities.

\subsection{Purpose}
The SDD shows how the smart email software system will be structured to satisfy the requirements identified in the conceptual design document and software requirements specification. It is a translation of requirements into a description of the software
structure, software components, interfaces, and data necessary for the implementation phase. In essence, the SDD is a detailed blueprint for the implementation activity.

\subsection{Scope}
\subsection{Definitions and acronyms}

\section{References}

\section{Decomposition description}
This project is being designed using an incremental approach. There are three primary stages to the design development which consists of Phase 1 (Classification Phase), due March 1, 2012, Phase 2 (Summarization Phase), due the week of April 1, 2012 and Phase 3 (Web Service Phase), due the week of May 1, 2012.

\subsection{System decomposition}
The system is divided into main modules as follows:
\begin{my_itemize}
\item Classification Module;
\item Summarization Module;
\item Web Service Moudle.
\end{my_itemize}
Each of these modules will be described in details in the upcoming subsections.

\subsubsection{Classification Module}

\begin{my_itemize}
  \item ClassificationManager class
  \begin{my_desc}
   \item[Purpose] test
   \item[Function] test
  \end{my_desc}
  \item PreprocessingManager class
  \item FilterManager class
  \item FilterCreatorManager class
  \item Instances class
  \item Instance class
  \item Attribute class
\end{my_itemize}

\subsubsection{Summarization Module}
\subsubsection{Web Service Module}

\subsection{Data decomposition}
\subsubsection{IMAP data access object}
\subsubsection{File system data access object}

\section{Dependency description}
\subsection{Intermodule dependencies}
\subsection{Data dependencies}

\section{Interface description}
\subsection{Module interface}
\subsubsection{Classification module interface}
\begin{my_itemize}
  \item Classifier class
  \begin{my_desc}
   \item[Purpose] All schemes for documents classification extend this class. Note that a classifier MUST either implement distributionForInstance() or classifyInstance() methods.
   \item[Function] The Classifier abstarct class defines the following functions:
	\begin{my_itemize}
		\item classifyInstance(Instance instance) : classifies a given instance;
		\item buildClassifier(Instances trainingData) : builds the classification model from the given set of training data;
		\item getClassifierByName(String name, String[] options) : returns an instance of the classifier given the classifier name;
		\item distributionForInstance(Instance instance): returns the destribution for each class attribute for the given instance.
	\end{my_itemize}
  \end{my_desc}

  \item Preprocessor interface
  \begin{my_desc}
   \item[Purpose] Performs some pre-processing actions on the email before classifiaction. Pre-processing may include removing stop words and stemming. 
   \item[Function] The Preprocessor interface defines the following function:
	\begin{my_itemize}
	\item process(Email email): Performs some pre-processing action on the given email such as stemming or removing stop words.
	\end{my_itemize}
  \end{my_desc}

  \item Filter class
  \begin{my_desc}
   \item[Purpose] Defines an abstract class for filtering an email. A filter is used to extract a set of features from the email such as the email sender, email label and word frequencies in an email.
   \item[Function] The filter abstract class defines the following set of functions:
	\begin{my_itemize}
	\item makeFeatureInstance(Email email): creates a feature instance from the given email.
	\item getAttributes(): returns the set of attributes for the given filter.
	\end{my_itemize}
  \end{my_desc}

  \item FilterCreator interface
  \begin{my_desc}
   \item[Purpose] Defines an interface for creating a filter.
   \item[Function] The FilterCreator interface defines the following function:
	\begin{my_itemize}
	\item createFilter(List<Email> emails): creates a filter given the list of emails.
	\end{my_itemize}
  \end{my_desc}

\end{my_itemize}

\subsubsection{Summarization moudle interface}
\subsubsection{Web service moudle interface}

\section{Detailed design}

\subsection{Classification module detailed design}

\begin{my_itemize}
  \item ClassificationManager class
  \item Classifier class
  \item PreprocessingManager class
  \item Preprocessor interface
  \item FilterManager class
  \item Filter class
  \item FilterCreatorManager class
  \item FilterCreator interface
  \item Instances class
  \item Instance class
  \item Attribute class
\end{my_itemize}


\subsection{Summarization module detailed design}
\subsection{Web service module detailed design}
\subsection{Data detailed design}

\begin{my_itemize}
  \item IMAP Data Access Object
  \item FileSystem Data Access Object
\end{my_itemize}


\end{document}