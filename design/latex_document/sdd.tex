\documentclass[12pt]{article}
\usepackage{multirow}
\usepackage[pdftex]{graphicx}
\usepackage{float}


\newenvironment{my_itemize}
{\begin{itemize}
  \setlength{\itemsep}{0cm}
  \setlength{\parskip}{0cm}}
{\end{itemize}}
\newenvironment{my_enumerate}
{\begin{enumerate}
  \setlength{\itemsep}{0cm}
  \setlength{\parskip}{0cm}}
{\end{enumerate}}
\newenvironment{my_desc}
{\begin{description}
  \setlength{\itemsep}{0cm}
  \setlength{\parskip}{0cm}}
{\end{description}}


\begin{document}

\begin{titlepage}
\vspace{-1.5cm}
\begin{center}
\includegraphics[width=2cm]{Logo_Alexandria_University.jpg}\\
\vspace{1cm}
\textbf{\large ALEXANDRIA UNIVERSITY} \\
\textbf{FACULTY OF ENGINEERING} \\
{\small  COMPUTER AND SYSTEMS ENGINEERING DEPARTMENT}

\vspace{2.5cm}
\textbf{\LARGE Software Design Description Document}\\
\textbf{\small Smart Email Automatic Email Classification and Summarization}\\
\vspace{1cm}
{ Ahmed El-Sharkasy, Ahmed Kotb, Amr Nabil, Mohammad Kotb, Moustafa Mahmoud }
\end{center}

\vspace{1ex}
\textbf{Supervisors:} Prof. Dr. Mohamed Abou-gabal, Dr. Mustafa ElNainay
\end{titlepage}

\newpage
\tableofcontents
\newpage

\section{Introduction}

This software design description document is a model of the smart email software system to be created. The model provides the precise design information needed for planning, analysis, and implementation of the software system. The document represents a partitioning of the system into design entities and describe the important properties and relationships among those entities.

\subsection{Purpose}
The SDD shows how the smart email software system will be structured to satisfy the requirements identified in the conceptual design document and software requirements specification. It is a translation of requirements into a description of the software
structure, software components, interfaces, and data necessary for the implementation phase. In essence, the SDD is a detailed blueprint for the implementation activity.

\subsection{Scope}
\subsection{Definitions and acronyms}

\section{References}

\section{Decomposition description}
This project is being designed using an incremental approach. There are three primary stages to the design development which consists of Phase 1 (Classification Phase), due March 1, 2012, Phase 2 (Summarization Phase), due the week of April 1, 2012 and Phase 3 (Web Service Phase), due the week of May 1, 2012.

\subsection{System decomposition}
The system is divided into main modules as follows:
\begin{my_itemize}
\item Classification Module;
\item Summarization Module;
\item Web Service Moudle.
\end{my_itemize}
Each of these modules will be described in details in the upcoming subsections.

\subsubsection{Classification Module}

\begin{my_itemize}
  \item ClassificationManager class
  \begin{my_desc}
    \item[Purpose] Used to manage and control the classification process flow
      with the dataset and classification algorithms.
    \item[Functions] TODO
  \end{my_desc}
  \item PreprocessingManager class
  \begin{my_desc}
    \item[Purpose] Used to manage preprocessing phase by applying selected data preprocessors
      to given raw emails.
    \item[Functions] \hfill
    \begin{my_itemize}
      \item apply(email:Email) used to apply the selected preprocessors on a given email.
    \end{my_itemize}
  \end{my_desc}

  \item FilterManager class
  \begin{my_desc}
    \item[Purpose] Used to manage feature extraction phase from preprocessed emails by applying
      specified set of filters.
    \item[Functions] \hfill
    \begin{my_itemize}
      \item getInstances(emails:List<Email>) given a set of emails returns a prepared dataset (Instances object)
      \item getInstance(email:Email) given an email return a prepared feature vector (Instance object)
    \end{my_itemize}

  \end{my_desc}

  \item FilterCreatorManager class
  \begin{my_desc}
    \item[Purpose] Used to manage all Filter creators to create specified filters given a list of emails
    \item[Functions] \hfill
    \begin{my_itemize}
      \item getFilters(emails:List<Email>) create all supported filters using given emails
    \end{my_itemize}

  \end{my_desc}


  \item Instances class
  \begin{my_desc}
    \item[Purpose] Used to gather all instances of the emails and treated as the classifier
      dataset.
    \item[Fuctions] \hfill
    \begin{my_itemize}
      \item add(instance:Instance) add new instance to the dataset.
      \item getAttribute(index:int) get Attribute at given index.
      \item getAttribute(name:String) get Attribute with the given name.
      \item checkInstance(instance:Instance) check if the given instance is in the dataset or not.
      \item getClassAttribute() return the class attribute of the dataset.
    \end{my_itemize}

  \end{my_desc}

  \item Instance class
  \begin{my_desc}
    \item[Purpose] Used to encapsulate the feature vector of a certain email.
    \item[Functions] \hfill
    \begin{my_itemize}
      \item getClassIndex() returns the index of the class attribute
      \item setClassIndex(index:int) set the index of the class attribute
      \item isMissingClass() check if the feature vector contains value of the class attribute or not
      \item getValue(index:int) return the value for the ith attribute
    \end{my_itemize}

  \end{my_desc}

  \item Attribute class
  \begin{my_desc}
    \item[Purpose] Used to encapsulate the dataset attribute with its type
    \item[Functions] TODO
  \end{my_desc}

\end{my_itemize}

\subsubsection{Summarization Module}
\subsubsection{Web Service Module}

\subsection{Data decomposition}
\subsubsection{IMAP data access object}
\subsubsection{File system data access object}

\section{Dependency description}
\subsection{Intermodule dependencies}
\subsection{Data dependencies}

\section{Interface description}
\subsection{Module interface}
\subsubsection{Classification module interface}
\begin{my_itemize}
  \item Classifier class
  \item Preprocessor interface
  \item Filter class
  \item FilterCreator interface
\end{my_itemize}

\subsubsection{Summarization moudle interface}
\subsubsection{Web service moudle interface}

\section{Detailed design}

\subsection{Classification module detailed design}

\begin{my_itemize}
  \item ClassificationManager class
  \item Classifier class
  \item PreprocessingManager class
  \item Preprocessor interface
  \item FilterManager class
  \item Filter class
  \item FilterCreatorManager class
  \item FilterCreator interface
  \item Instances class
  \item Instance class
  \item Attribute class
\end{my_itemize}


\subsection{Summarization module detailed design}
\subsection{Web service module detailed design}
\subsection{Data detailed design}

\begin{my_itemize}
  \item IMAP Data Access Object
  \item FileSystem Data Access Object
\end{my_itemize}


\end{document}